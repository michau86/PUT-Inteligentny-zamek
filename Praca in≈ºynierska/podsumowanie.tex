% !TeX spellcheck = pl_PL
\newpage
\section{Podsumowanie} \label{sec:podsumowanie}
	Podsumowując prace wykonaną podczas tworzenia pracy dyplomowej zrealizowano wszystkie punkty przewidziane w zadaniach szczegółowych umieszczonych w kartach prac. Wykonano aplikację mobilną z usprawnieniami względem prac wykonanych w ramach przedmiotów. Dopracowano i ujednolicono interfejs graficzny, czyniąc tym aplikację przyjazną dla użytkownika. Poza aplikacją mobilną rozszerzono funkcjonalności serwera o obsługę infrastruktury klucza publicznego, stronę internetową oraz szeroko pojęte zarządzanie kontami użytkowników. Zupełnie nową możliwością systemu jest zliczanie osób. \newline
	Bazując na pracy wykonanej w ramach wcześniejszych przedmiotów wykonano w pierwszej kolejności zgodnie z~metodyką kaskadową  planowanie oraz analizę systemu pod kątem praktycznego wykorzystania w postaci między innymi przypadków użycia. Następnie W trakcie etapu tworzenia projektu znaczna uwaga została skupiona na  przewidzeniu wszystkich możliwych  elementów potrzebnych do utworzenia oraz implementacji zgodnej z dobrymi praktykami programowania.
	
	W dalszej fazie opisanej w dokumencie została wykonana implementacja poszczególnych modułów systemu. W~trakcie realizacji z powodów ograniczeń sprzętowych urządzenia Raspberry Pi. okazało się że moc obliczeniowa dla oprogramowania sterującego zamkiem oraz kodu odpowiedzialnego za zliczanie osób w pomieszczeniu jest niewystarczająca w stosunku do tego co oferuje te urządzenie.
	
	Po procesie implementacji wraz z poprawą wykrytych na bieżąco błędów przeprowadzony został wewnętrzny audyt  systemu. Był on oparty o metodologię OWASP. Audyt ten został szczegółowo opisany w rozdziale Bezpieczeństwo (rozdział \ref{sec:bezpieczenstwo}).
	
	Po wykonaniu tego audytu odkryte zostały potencjalne nieprzewidziane zagrożenia oraz zostały zaproponowane potencjalne możliwości naprawy ich. Następnie przeprowadzony został proces testowania podczas którego odkryto problem z komunikacją pomiędzy urządzeniem sterującym oraz aplikacją mobilną która powoduje brak informowania użytkownika o fakcie uzyskania błąd odrzucenia dostępu do pomieszczenia podczas otwierania zamka. W fazie testowania użyty został certyfikat SSL z niezaufanego źródła, wygenerowany przy pomocy biblioteki ''Werkzug''. Żeby moduły akceptowały połączenia z niezaufanego źródła zostały zaimplementowane funkcje zezwalające na nie. Przed wdrożeniem projektu należałoby zapewnić prawdziwy certyfikat SSL z zaufanego źródła oraz wyłączyć w modułach możliwość łączenia z niezaufanymi źródłami.               
	\subsection{Dalsze perspektywy rozwoju projektu}
	W ramach dalszych perspektyw rozwoju systemu \NazwaSys w pierwszej kolejności należałoby wyeliminować zagrożenia odkryte podczas przeprowadzonego audytu oraz  wyeliminować problemy zauważone w fazie testowania. 
	W dalszych krokach można rozszerzyć pracę dyplomową o:
	\begin{itemize*}
	\item  obsługę urządzeń z systemami operacyjnymi takimi jak iOS czy Tize,
	\item dodanie możliwości korzystania z aplikacji przeznaczonej na smartwatche, 
	\item zaprojektowanie oraz implementacje modułów ekonomicznych z wykorzystaniem mikroprocesora STM,
	\item rozszerzenie funkcjonalności systemu o funkcje związane z tematem internetem przedmiotów.
	 
	\end{itemize*}
	
 
	 