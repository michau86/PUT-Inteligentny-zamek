\newpage\section{Wybór technologii informatycznych} \label{sec:technologie}
\subsection{Urządzenie sterujące}

\subsection{Aplikacja serwera}
Aplikacja serwerowa została stworzona przy pomocy zintegrowanego środowiska programistycznego PyCharm w wersji 2017.1.3. Technologie użyte w aplikacji serwerowej były następujące:
\begin{itemize}
	\item python w wersji 2.7
	\item framework Django
	\item MySQl 
	\item Ajax
	\item jQuery
	\item JSON
\end{itemize}
Ponadto oprogramowanie serwera było testowane przy pomocy narzędzia XAMMP w wersji 3.2.2 które emulowało środowisko apache oraz baze danych MySQl.
\subsection{Aplikacja mobilna}
Aplikacja mobilna została stworzona przy pomocy zintegrowanego środowiska programistycznego Android Studio w wersji 3.0 wraz z zintegrowanym emulatorem Genymotion w darmowej wersji. Języki użyte w aplikacji były nastepujące:
\begin{itemize}
	\item Java 
	\item Kotlin
	\item XML
	\item JSON
\end{itemize}
Cała aplikacja ponaddto została napisana w oparciu o wzorzec architektoniczny Model View Presenter. 
\subsection{Moduł zliczania osób}
\subsection{System kontroli wersji}
Podczas tworzenia naszej poracy użyliśmy systemu kotroli wersji GIT wraz z  oprogramowaniem dekstopowym przeznaczonym do środowiska windows o anzwie GitHub Dekstop.
\subsection{Prowadzenie dokumentacji}
 Dokumentacje prowadziliśmy w języku LaTeX przy pomocy oprogramowania TexStudio. 
