% !TeX spellcheck = pl_PL
\newpage\section{Wybór technologii informatycznych} \label{sec:technologie}
Prezentowany projekt ze względu na rozbudowaną konstrukcję, składa się również z wielu technologii informatycznych. Dominującymi językami programowania są Python oraz Java (z elementami języka Kotlin). Wybór oraz uzasadnienie decyzji zostało przedstawione w poniższych punktach.

\subsection{Urządzenie sterujące}
Urządzenie sterujące utworzone zostało na mikrokomputerze Raspberry Pi 3, ze względu na szereg interfejsów dostępnych na płytce. Inne rodzaju mikrokontrolery np. typy STM32 umożliwiają podłączenie takich urządzeń jak bluetooth oraz karta sieciowa Wifi, lecz jest to wówczas zintegrowany układ (celem pracy dyplomowej nie było tworzenie płytki PCB pozwalającej zintegrować układy) oraz drugim powodem jest wydajność i elastyczność programów. 

Podczas projektowania urządzenia sterującego zastosowano następujące technologie:
\begin{itemize*}
	\item język programowania Python w wersji 2.7.13 wraz z bibliotekami PyCrypto, MySQLdb oraz BlueZ,
	\item format zapisu danych JSON,
	\item biblioteka RPI.GPIO, aby uzyskać dostęp do interfejsu wejścia/wyjścia.
\end{itemize*}

Programem wykorzystywanym do implementacji programu jest PyCharm w wersji 2017.2.4. Wybór technologii oraz oprogramowania związany był z wysokim stopniem integracji urządzenia Raspberry Pi z językiem Python. Środowisko Pycharm zostało wybrane ze względu na zaawansowane funkcjonalności wspomagające tworzenie kodu o wysokiej czytelności oraz IDE udostępnia funkcję InteliSense (podpowiedzi pod względem wartości, pól klas, procedur).

\newpage
\subsection{Aplikacja serwera}
Aplikacja serwerowa została stworzona przy pomocy zintegrowanego środowiska programistycznego PyCharm w wersji 2017.1.4. Technologie użyte w aplikacji serwerowej były następujące:
\begin{itemize*}
	\item język programowania Python w wersji 2.7.13,
	\item framework Django,
	\item MySQl,
	\item JSON.
\end{itemize*}

Ponadto oprogramowanie serwera było testowane przy pomocy narzędzia XAMMP w wersji 3.2.2 które emulowało środowisko apache oraz baze danych MySQl.

\newpage
\subsection{Aplikacja mobilna}
Aplikacja mobilna została stworzona przy pomocy zintegrowanego środowiska programistycznego Android Studio w wersji 3.0 wraz z zintegrowanym emulatorem Genymotion w darmowej wersji. Języki użyte w aplikacji były nastepujące:
\begin{itemize}
	\item Java 
	\item Kotlin
	\item XML
	\item JSON
\end{itemize}
Cała aplikacja ponaddto została napisana w oparciu o wzorzec architektoniczny Model View Presenter. 

\newpage
\subsection{Moduł zliczania osób}

\newpage
\subsection{System kontroli wersji}
Podczas tworzenia naszej poracy użyliśmy systemu kotroli wersji GIT wraz z  oprogramowaniem dekstopowym przeznaczonym do środowiska windows o anzwie GitHub Dekstop.

\newpage
\subsection{Prowadzenie dokumentacji}
 Dokumentacje prowadziliśmy w języku LaTeX przy pomocy oprogramowania TexStudio. 
