% !TeX spellcheck = pl_PL
\newpage
\section{Bezpieczeństwo systemu \textsl{\NazwaSys}} \label{sec:bezpieczenstwo}
Rozdział dotyczy tematyki bezpieczeństwa prezentowanego systemu podsumowując fazę projektową oraz wykonawczą. Podstawowymi mechanizmami zabezpieczenia modułów są techniki kryptograficzne opisane w punkcie \ref{sec:techniki kryptograficzne}. Następnie zostaną one zweryfikowane pod względem podatności wg listy najczęściej popełnianych błędów programistycznych (OWASP Top 10) oraz innych losowych sytuacji. Na koniec tego rozdziału zostaną zaproponowane rozwiązania pozwalające na poprawę bezpieczeństwa systemu.

\subsection{Techniki kryptograficzne [\StudentB]}\label{sec:techniki kryptograficzne}
W systemie \textsl{\NazwaSys} zaimplementowano szereg funkcji kryptograficznych. Komunikacja pomiędzy modułami wymienionymi w punkcie \ref{Komunikacja serwer}, odbywa się przy pomocy Web API. Sama transmisja danych oparta jest o~protokół Https. Po stronie serwera przy pomocy paczek ''pySSl'' oraz ''Werkzug'' została zaimplementowana funkcja SSL dla wszystkich API włącznie z stroną internetową. Z racji braku posiadania własnego certyfikatu SSL w~projekcie tym został wygenerowany przy pomocy paczki ''Werkzug'' przykładowy certyfikat. Z~racji tego że ten urząd nie jest rozpoznawalny wyświetlają się komunikaty na~stronie internetowej o niebezpieczeństwie w postaci niezaufanego certyfikatu. W~aplikacji mobilnej na~czas korzystania z niego została zaimplementowana opcja zezwalająca na używanie certyfikatów z niezaufanego źródła. Podczas wdrażania projektu w prawdziwych warunkach zalecamy wykorzystanie prawdziwego certyfikatu.

Funkcje kryptograficzne użyte w aplikacji mobilnej należy rozdzielić na~dwie sekcje. Pierwszą z~nich są funkcje używane podczas przechowywania danych użytkownika. Wszystkie newralgiczne informacje są szyfrowane w celu polepszenia poufności w systemie. Dane przechowywane w pamięci telefonu szyfrowane są funkcją kryptograficzną AES w trybie blokowym CBC. W przypadku aplikacji mobilnej korzystaliśmy z gotowej implementacji zawartej w bibliotece ''crypto'' przeznaczonej dla języka Java. Dane szyfrowane są trzema rodzajami haseł i~tak hasło użytkownika przechowywane jest w pamięci telefonu (SharedPreferences) w postaci zaszyfrowanego tekstu, który jest szyfrowany hasłem zaszytym w~implementacje aplikacji. Klucz szyfrujący oraz token szyfrowane są za pomocą hasła użytkownika. Podczas eksportu klucza szyfrującego wraz z~certyfikatem klucza szyfrującego, które łączone są w jeden plik   występuje dodatkowe szyfrowanie hasłem które użytkownik wpisze. Zapewnia to zwiększoną poufność oraz poprawia bezpieczeństwo podczas przenoszenia pliku.                

Drugą sekcją są funkcje kryptograficzne użyte podczas generowania klucza szyfrującego oraz używania go do podpisów. Algorytmem użytym do tego jest RSA. Sygnatura wykorzystywana w tym algorytmie to SHA-256. Do implementacji tych funkcji skorzystaliśmy z gotowej biblioteki ''security''. Klucze te są podstawą kryptograficzną naszego projektu oraz implementacji PKI. Generowane są one po stronie aplikacji mobilnej i z pary kluczy prywatny publiczny tylko publiczny jest przesyłany do serwera, i dzięki temu oraz polityce przechowywania tego klucza uniemożliwiamy przejęcie tego klucza przez osoby niepowołane. PKI umożliwia jednoznacznie określić, kto jest nadawcą wiadomości oraz czy nie została naruszona integralność podczas przesyłu.

Aplikacja zarządzająca zamkiem korzysta z funkcji kryptograficznych takich jak RSA oraz SHA-256. Wykorzystuje ona je w procesie weryfikacji certyfikatu użytkownik wysłanego z urządzenia mobilnego. Do implementacji tych funkcji wykorzystano gotową bibliotekę ''PyCrypto''

\newpage
\subsection{Podatności systemu (OWASP Top 10) [\StudentA]} \label{sec:OWASP}
W punkcie tym zostaną opisane podatności systemu na zagrożenia według metodologi OWASP TOP 10 z 2017 roku. Wszystkie zagrożenia zostały wypisane poniżej wraz z opisem, czy dane zagrożenie występuje w prezentowanym systemie oraz wykazaniem podatności (lub nie).
\begin{enumerate*}
	\item SQL Injection 
	
	 Podatność nie występuje w prezentowanym systemie ze względu na zabezpieczenie w serwerze Web API pod kątem tego zagrożenia, które zostało szerzej opisane w punkcie \ref{sec:apk serw}.
	\item złamanie uwierzytelnienia 
	
	 Zagrożenie nie występuje w systemie ze względu na walidację hasła przeprowadzaną podczas rejestracji użytkownika oraz funkcji zmiany hasła przy pomocy wyrażenia regularnego które wymusza odpowiednie parametry hasła. Hasło to musi spełniać następujące warunki. Posiadać musi przynajmniej jedną dużą literę, jedną małą literę,  cyfrę oraz znak specjalny.
	\item narażenie na ekspozycje czułych danych 
	
	 Nie występuje ono dzięki polityce generowania oraz przechowywania wrażliwych danych,  zabezpieczenie ekspozycji tych informacji przy pomocy szyfrowanego połączenia między modułami oraz zabezpieczenia aplikacji serwerowej pod względem podatności na ataki uwzględniające przejecie wrażliwych danych.
	\item zewnętrzne encje XML 
	
	 System \NazwaSys \space nie jest podatny na tego typu zagrożenia ze względu na nieużywanie tego formatu do przechowywania danych. Obecnie podczas przesyłu informacji oraz przetrzymywania ich poza bazą danych używamy formatu JSON.
	\item zepsuta kontrola dostępu 
	
	 Nie wykryto takiego zagrożenia. Wszystkie funkcje uwierzytelniające są zabezpieczone oraz przetestowane pod kątem poprawności wykonywania oraz kontroli dostępu do części systemu wymagających konkretnych uprawnień.
	 \item Błędna konfiguracja zabezpieczeń
	 
	 Obecnie w celach testowych istnieje taka podatność. Znajduje się ona w aplikacji mobilnej w  postaci umożliwienia nawiązywania połączenia Https z serwerem który posiada certyfikat SSL z niezaufanego źródła. Zostało to wprowadzone w celach testowych. Rozwiązanie to zostało opisane w punkcie \ref{sec:techniki kryptograficzne} wraz z zaleceniami dotyczącymi poprawy bezpieczeństwa w warunkach docelowych. Ponadto nie zostały wykorzystane dobre praktyki związane z uwierzytelnieniem użytkownika w postaci stosowania soli do wygenerowania funkcji skrótu z hasła.
	 
	 \item podatność XSS
	 
	 Istnieje możliwość podatności XSS typu reflected w momencie kiedy administrator kliknie w niezaufany link. Kluczową kwestią w tym wypadku jest odpowiednie przeszkolenie kadry ad ministrującej danym serwerem by nie dopuścić do ataku opierającego się w sporej mierze na wykorzystaniu socjotechniki. Ponadto brak podstawowych zabezpieczeń na tego typu ataki w postaci filtracji danych przesyłanych przez użytkownika. 
	 
	 \item Niezabezpieczona deserializacja
	 
	 Nie istnieje takie zagrożenie w systemie. Nie ma możliwości zmiany żadnego fragmentu kodu przez osoby trzecie oraz fakt przejęcia i modyfikacji danych nie umożliwi w żaden sposób uzyskanie dostępu do jakiejkolwiek części systemu.
	 
	 \item  Używanie komponentów z znanymi podatnościami
	 
	 Podatność ta została ograniczona dzięki dbaniu o wykorzystywanie najnowszych funkcji oraz aktualizowanie na bieżąco wszystkich komponentów znajdujących się w modułach projektu.
	 
	 \item Niewystarczające monitorowanie 
	 
	 Zagrożenie te występuje w systemie. Nie ma żadnego zdarzenia monitorującego błędnych prób logowania do systemu oraz nieautoryzowanych prób wykonania poszczególnych API na serwerze. Ponadto nie zostały przeprowadzone testy penetracyjne oraz nie zostały zaimplementowane funkcje ostrzegające o potencjalnych atakach w czasie rzeczywistym\cite{OWASP}.
\end{enumerate*}

\newpage
\subsection{Inne zagrożenia występujące w systemie [\StudentB]}

Oprócz zagrożeń wykrytych w podpunkcie \ref{sec:OWASP} w systemie zostały wykryte następujące zagrożenia. W~momencie kradzieży urządzenia na którym użytkownik jest zalogowany można uzyskać z jego konta dostęp do pomieszczeń.
 
 Ponadto z powodu braku zabezpieczeń funkcji newralgicznych takich jak generowanie certyfikatu czy akceptowanie rejestracji przez użytkownika zalogowanego z uprawnieniami administratora, w momencie kradzieży urządzenia istnieje możliwości wygenerowania dostępu do każdego z możliwych pomieszczeń oraz przyznanie tychże uprawnień dla dowolnego z użytkowników systemu. 
 
 W przypadku utraty klucza szyfrującego przez użytkownika nie ma przewidzianej żadnej funkcji umożliwiającej przez administratora metody pozwalającej na odzyskanie certyfikatu klucza szyfrującego potrzebnego do wygenerowania nowego klucza szyfrującego.
 
\subsection{Możliwości poprawy bezpieczeństwa systemu [\StudentA]}

Dzięki przeprowadzonej analizie pod względem metodologi OWASP oraz weryfikacji pod kątem innych zagrożeń została sporządzona lista możliwości poprawy systemu pod względem bezpieczeństwa. 
\begin{itemize*}
	\item zablokowanie możliwości połączenia http gdy certyfikat pochodzi z niezaufanego źródła,
	\item zastosowanie soli podczas uwierzytelniania użytkownika,
	\item  zastosowanie filtracji danych wprowadzonych przez użytkownika w formularzu logowania dla strony internetowej,
	\item poprawa API pod względem monitorowania błędnych prób autoryzacji w systemie,
	\item  poprawa API pod względem nieautoryzowanych prób dostępu do poszczególnych API,
	\item wykonanie testów penetracyjnych wraz z ewentualnym wyeliminowaniem zagrożeń wykrytych,
	\item zaimplementowanie funkcji ostrzegających o atakach przeszukujących lub typu DDoS administratora systemu w czasie rzeczywistym,
	\item umożliwienie szybszej możliwości zgłoszenia unieważnienia konta, 
	\item wprowadzenie funkcji otwierania zamka oraz dostępu do metod administratora po autoryzacji czytnikiem linii papilarnych,
	\item zabezpieczenie metod wymagających uprawnień administratora poprzez zastosowanie wymogu wprowadzania hasła autoryzującego,
	\item przygotowanie narzędzia dla administratora systemu ułatwiającego przywracanie klucza dostępowego dla użytkownika. 
\end{itemize*}

 
