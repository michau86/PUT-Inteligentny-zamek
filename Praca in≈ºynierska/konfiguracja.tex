\def\TytulPolski    {Projekt i wykonanie systemu kontroli ruchu i zarządzania dostępem do pomieszczeń}
\def\TytulAngielski {Design and implementation of movement control and access to spaces managment system }
 
% Nazwa systemu informatycznego
\def\NazwaSys {\textit{Inteligentny zamek}}

\def\Promotor    {dr inż. Ewa Idzikowska}
\def\StudentA     {Maciej Marciniak}
\def\AlbumA       {121996}
\def\StudentB     {Damian Filipowicz}
\def\AlbumB       {122002}

%Kierunek:
\def\Kierunek {Informatyka}
 
%Specjalnosc:
\def\Specjalnosc {Bezpieczeństwo systemów informatycznych} 

%Poziom studiów: 
\def\PoziomStudiow {I stopnia}
%\def\PoziomStudiow {II stopnia}

%Forma studiów:
\def\FormaStudiow {stacjonarne}
%\def\FormaStudiow {niestacjonarne}


% Kierunek:      Informatyka - studia stacjonarne I stopnia
%                Informatyka - studia niestacjonarne I stopnia
% Specjalność:   Technologie informatyczne
%        

% Zmiana nazw w opisie tabel i rysunków
\AtBeginDocument{
    \renewcommand*{\tablename}{Tabela}
    \renewcommand*{\figurename}{Rys.} 
		% TO DO: Spis tablic -> Spis tabel
}
\setcounter{secnumdepth}{4}

\titleformat{\paragraph}
{\normalfont\normalsize\bfseries}{\theparagraph}{1em}{}
\titlespacing*{\paragraph}
{0pt}{3.25ex plus 1ex minus .2ex}{1.5ex plus .2ex}
\setlength{\parindent}{25pt}
\setlength{\parskip}{5pt}
\frenchspacing
\newcommand{\linia}{\rule{\linewidth}{0.4mm}}
\newcommand{\tablinia}{\newline \linia \newline}   
\counterwithin{figure}{section}
\counterwithin{table}{section}

\hypersetup{
	colorlinks=true,
	linkcolor=black,
	filecolor=magenta,      
	urlcolor=cyan,
}


% Jeżeli kody programów zawierają polskie litery należy dodać odpowiedni znak poniżej
\lstset{literate=%
    {ż}{{\.z}}1
    {ą}{{\c{a}}}1
    {ę}{{\c{e}}}1
		{ó}{{\'o}}1
		{ć}{{\'c}}1
		{ś}{{\'s}}1
}
        


