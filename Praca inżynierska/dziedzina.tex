% !TeX spellcheck = pl_PL
% 
\newpage
\section{Opis dziedziny przedmiotowej pracy}\label{sec:dziedzina}
Rozdział zawierać będzie objaśnienia używanych zwrotów i pojęć umożliwiających poprawne interpretowanie dalszych tekstów. Następnie opisany zostanie stan wiedzy związanej z tematyką pracy, to znaczy omówienie wybranych rozwiązań systemów tak zwanych inteligentnych zamków. W zestawieniu porównane zostaną również zaproponowane w pracy dyplomowej rozwiązania.
\subsection{Pojęcia i definicje}\label{sec:Pojęcia i definicje}
W dokumencie tym posługiwać się będziemy następującymi pojęciami:
\begin{itemize*}
	\item {Klucz dostępowy} --- jest to  klucz określający dostęp do pomieszczenia dla~użytkownika w konkretnych dniach oraz godzinach,
	\item {Klucz szyfrujący} 
	--- jest to klucz prywatny \footnote{ klucz prywatny szerzej został opisany w rozdziale \ref{sec:Projekt PKI}} wygenerowany podczas tworzenia certyfikatu klucza szyfrującego. Używany jest on do szyfrowania wiadomości wysyłanej z aplikacji mobilnej do urządzenia sterującego,
	\item {Klucz deszyfrujący }
	--- jest to klucz publiczny \footnote{ klucz publiczny szerzej został opisany w rozdziale \ref{sec:Projekt PKI}} wygenerowany podczas tworzenia certyfikatu klucza szyfrującego. Używane jest on do odszyfrowania wiadomości wysyłanej z aplikacji mobilnej do urządzenia sterującego,
	\item {Inteligentny zamek}
	--- system obsługujący otwieranie elektrozamka bądź serwomechanizmu,
	\item {Konto}
	--- reprezentacja użytkownika w systemie za pomocą takich danych jak login hasło, imię, nazwisko.
	\item {Administrator}
	--- jest to fizyczna osoba posiadającą dla swojego konta uprawnienia administratora co wiąże się z pełnym dostępem do systemu,
	\item {Certyfikat dostępowy}
	--- jest to certyfikat przechowujący informacje takie jak dane użytkownik, do jakiego pomieszczenia oraz w jakich dniach i~godzinach	ma dostęp
	\item {Certyfikat klucza szyfrującego}
	-- jest to certyfikat przechowujący dane o~użytkowniku, ważności klucza szyfrującego oraz sam klucz deszyfrujący.
\end{itemize*}

\subsection{Stan wiedzy}\label{sec:Stan wiedzy}
Przed przystąpieniem do projektu wykonaliśmy rozeznanie w około systemów zbliżonych do naszego, który na dany moment były produkowane. I~tak~doszliśmy do wniosku, że wszystkie systemy inteligentnych zamków wykonane przez uznane firmy, takie jak Gerda Lock, czy Danalock zostały wykonane typowo dla użytku domowego a nie tak jak nasz projekt inżynierski, który jest przeznaczony do zarządzania w budynkach o większej złożoności, takich jak biurowce, z różnym stopniem dostępu. Opis wraz z porównaniem poszczególnych systemów znajduje się w tabelach poniżej.

Tabela \ref{tab:porownanie1} zawiera porównanie firm pod względem otwierania zamka. Już w pierwszym wierszu można zauważyć, indywidualne zastosowanie systemów, ponieważ każdy z prezentowanych systemów, poza firmą August, nie~oferuje obsługi wielu urządzeń z poziomu jednej aplikacji. Proponowane przez nas~rozwiązanie umożliwia skalowalność systemu oraz wprowadzanie różnorodności w~zarządzaniu dostępem. Następne funkcjonalności wprowadzają zagrożenia lub zwiększają podatności sytemu na ataki hakerskie. Weźmy pod uwagę otwieranie dowolnego zamka z dowolnego miejsce przez stronę WWW, umożliwia taka funkcjonalność zdalne sterowanie dostępem w całym budynku. Skutkuje to brakiem kontroli, którą nie powoduje znacznych korzyści dla celów biznesowych dla, których nasz system jest dedykowany.
\newpage
	\begin{longtable}[!ht]{|m{6cm}|m{1.4cm}|m{1.4cm}|m{1.7cm}|m{2cm}|m{2cm}|} 
		\caption{Tabela porównania otwierania zamków}
		\label{tab:porownanie1}\\
		\hline	
		Funkcja & NUKI\footnote{ strona producenta z opisem zamka firmy NUKI: \href{https://nuki.io/en/smart-lock/}{https://nuki.io/en/smart-lock/}} & August\footnote{ Strona producenta z opisem zamka firmy August: \href{http://august.com/keyless-entry/ }{http://august.com/keyless-entry/}} & DanaLock\footnote{ Strona producenta z opisem zamka firmy DanaLock: \href{https://danalock.com/danalock-v3.html}{https://danalock.com/danalock-v3.html}} &
		 Gerda Lock \footnote{ Strona producenta z opisem zamka firmy Gerda Lock: \href{https://www.gerdalock.com/produkty/gerdalock}{https://www.gerdalock.com/produkty/gerdalock}} & 
		 Nasz system  \\	\hline
		 Zarządzanie wieloma zamkami z jednej aplikacji
		 & brak & tak & brak & brak & tak \\	\hline
		Otwieranie zamka przy pomocy strony WWW
		& brak & brak & tak & brak & brak \\	\hline
		Inne sposoby otwarcia zamka niż aplikacja
		& brak & brak & brak & tak & tak \\	\hline
		Automatyczne zamykanie zamka
		& brak & tak & tak & tak & tak \\	\hline
		Tryb otwierania zamka automatycznie
		& tak& brak & tak & tak & nie \\	\hline	
		Tryb otwierania zamka po zezwoleniu przyciskiem
		& brak & tak & tak & tak & tak \\		\hline
	\end{longtable}

 	Tabela \ref{tab:porownanie2} zawiera porównanie firm pod względem zasilania i montażu. Wszystkie systemy zawierały podstawowe wady eksploatacyjne, gdy system w~budynku zawierałby wiele urządzeń danego typu, to znaczy problem z zasilaniem. Nasz system oferuje podłączenie zasilania bezpośrednio z sieci, taka konfiguracja pozwala uodpornić system na konieczność wymiany baterii w każdym zamku. Gdyby jednak konieczne było zasilanie awaryjne, w sytuacji zaniku napięcia, możliwe jest zastosowanie UPSów, które podtrzymają urządzenia przez czas awarii.
 \begin{longtable}[!ht]{|m{5cm}|m{1.7cm}|m{1.7cm}|m{1.7cm}|m{2cm}|m{2.4cm}|} 
 	\caption{Tabela porównania zasilania i montażu}
 	\label{tab:porownanie2}\\
 	\hline	
 	Funkcja & NUKI & August &DanaLock & Gerda Lock & Nasz system \\	\hline
 	Zasilanie zewnętrzne (z sieci)	
 	& brak & brak & brak & brak & tak \\	\hline
	 Zasilanie bateryjne \newline (podstawowe/ awaryjne)	
	 & podstawowe & podstawowe &podstawowe & podstawowe & możliwe \newline awaryjne \\	\hline
 	Sposób montażu	
 	& nakładka na \linebreak zamek & nakładka na \linebreak zamek & nakładka na \linebreak zamek & nakładka na \linebreak zamek & nakładka na \newline zamek lub  \newline elektrozamek  \\	\hline
 \end{longtable}
 
Tabela \ref{tab:porownanie3} zawiera porównanie firm pod względem dziennika zdarzeń oraz powiadomień. Funkcjonalność naszego systemu zgodna jest z możliwościami firm NUKI oraz Gerda Lock. Pozostali dystrybutorzy nie udostępniają tak szczegółowych informacji na temat działa urządzeń.
\begin{longtable}[!ht]{|m{6cm}|m{1.4cm}|m{1.4cm}|m{1.7cm}|m{2cm}|m{2cm}|} 
	\caption{Tabela porównania powiadomień działania}
	\label{tab:porownanie3}\\
	\hline	
	& NUKI & August & DanaLock & Gerda Lock & Nasz system \\	\hline
	podgląd kto otworzył zamek
	& taki & brak & brak & tak & tak \\	\hline
	powiadomienie o otwarciu drzwi \newline (ogólnie i przez daną osobę)
	& brak & brak & brak & tak & tak  \\	\hline
	powiadomienie o nieautoryzowanych próbach otwarcia
	& tak & brak & brak & tak & tak  \\	\hline
\end{longtable}

\newpage
\subsection{Stan pracy wykonany w ramach zajęć \newline przedmiotowych} \label{sec:Stan pracy wykonany w ramach zajęć przedmiotowych}
W ramach zajęć projektowych oraz laboratoryjnych o nazwę Projekt Zespołowy prowadzonych z mgr. Michałem Apolinarskim oraz dr Ewą Idzikowską zostały wykonane następujące fragmenty systemu:
	
	Aplikacja mobilna została wykonana dla wersji androida minimum 4.4 KitKat w stopniu umożliwiającym podstawowe funkcjonalności, takie jak:
	\begin{itemize*}
		\item Logowanie użytkowników,
		\item Rejestracja użytkowników,
		\item Rejestracja wraz z tworzeniem certyfikatu klucza szyfrującego,
		\item Generowanie nowego certyfikatu,
		\item Pobieranie certyfikatów z serwera,
		\item Zarządzanie certyfikatami użytkownika,
		\item Zarządzanie prośbami o rejestracje,
		\item Wnioskowanie o certyfikat nowy.
	\end{itemize*}

		Dodatkowo zostało zaimplementowane gniazdka sieciowe do obsługi połączenia bluetooth oraz w każdym widoku, który korzystał z połączenia z serwerem, były napisane fragmenty kodu. Funkcje te oraz kod zostały napisane bez~uwzględnienia wzorców architektonicznych (wszystko dotyczące danego widoku było zawarte w jednej klasie), posiadały szereg błędów powodujących niestabilne działanie systemu oraz stosowały metody z systemu android. które były określane przez środowisko android studio, jako ''deprecated''\footnote{ Jeśli funkcja jest oznaczona jako deprecated to jest to informacja o tym, żę dana funkcja jest już przestarzała i może w nowszych modelach telefonów (dla nowszych systemów Android) nie działać}. Z racji pisania pod wersje systemu android 4.4 wygląd różni się od tego, który został zaimplementowany w pracy dyplomowej.

   Aplikacja serwerowa została wykonana w stopniu umożliwiającym podstawowe funkcje pozwalające na komunikację pomiędzy urządzeniami. Api zapewniało w minimalnym stopniu bezpieczeństwo. Funkcje serwera utworzone w ramach zajęć przedmiotowych:
   	\begin{itemize*}
   		\item Logowanie użytkowników,
   		\item Rejestracja użytkowników,
   		\item Zapisywanie nowego certyfikatu dostępu,
   		\item Udostępnianie certyfikatów dostępowych,
   		\item Pobieranie list certyfikatów, próśb o certyfikaty oraz rejestracji.
   	\end{itemize*}
   
	Wszystkie te funkcje zwracały odpowiednio pożądane dane, albo wartość ''Invalid'' co powodowało wyświetlanie komunikatów o błędach użytkownikowi bez rozróżniania powodu, np. awarii serwera (mylnie wysyłany komunikat ''Invalid'' zamiast komunikatu HTTP typu 500).
	
	Urządzenie sterujące zamkiem zostało napisane w języku Python i pozwalało na odbieranie danych z aplikacji mobilnej oraz posiadało funkcje odpowiedzialną za otwieranie zamka.

	W ramach przedmiotu ochrona danych prowadzone przez dr inż. Anne Grocholewską-Czuryło zostały zaimplementowane w systemie fragmenty PKI, takie jak:
	   \begin{itemize*}
	   	\item Format certyfikatu klucza szyfrującego,
	   	\item Generowanie nowego certyfikatu szyfrującego użytkownika,
	   	\item Blokowanie użytkowników oraz certyfikatów szyfrujących systemu.
	   \end{itemize*}
   
	 Funkcje te zostały napisane zarówno po stronie aplikacji mobilnej jak~i~aplikacji serwerowej. Ponadto po stronie androida został opracowany sposób przechowywania klucza prywatnego w formie zaszyfrowanego pliku hasłem użytkownika.
