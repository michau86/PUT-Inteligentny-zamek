% !TeX spellcheck = pl_PL
%\newpage\section{Dodatek} 
\newpage
\section*{Dodatki} \label{Dodatki}
\addcontentsline{toc}{section}{Dodatki}
%\newpage\textbf{Dodatki}\\
%\addcontentsline{toc}{section}{Dodatki}
\subsection*{Instalacja systemu \NazwaSys}
Instalacje systemu należy podzielić na 4 kroki. Pierwszym z nich jet instalacja serwera.  Żeby to wykonać należy zainstalować w podanej kolejności następujące elementy w systemie linux bąd"z Windows:
\begin{itemize*}
	\item Środowisko python 2.7 wraz z programem pip
	\item biblioteka pyCrypto zainstalowane przy pomocy polecenia pip -install pyCrypto
	\item  framework Django przy pomocy polecenia pip -install Django
	\item biblioteka MySQLdb przy pomocy polecenia pip -install MySQLdb
	\item  biblioteka pySSL przy pomocy polecenia pip -install pySSL
	\item biblioteka Werkzug przy pomocy polecenia pip - install Werkzug
\end{itemize*}
W celu uruchomienia serwera należy podac polecenie python manage.py runserver\_plus xxx.xxx.xxx.xxx:8080 --cert-file cert.crt gdzie w miejscu ''xxx.xxx.xxx.xxx'' należy wpisać adres IP serwera.

Kolejnym elementem instalacyjnym systemu jest?????

Ostatnim elementem potrzebnym do funkcjonowania projektu jest aplikacja mobilna. W celu zainstalowania jej potrzebny będzie telefon mobilny z wersją systemu operacyjnego android w wersji minimalnej 5.0 z włączoną funkcją programistyczną  W celu wgrania aplikacji na telefon należy przekopiować plik z rozszerzeniem apk do pamięci telefonu a następnie uruchomić go w urządzeniu mobilnym (należy wcześniej zezwolić na instalowanie oprogramowania z nieznanego "zródła).

  
\subsection*{Instrukcja użytkownika systemu \NazwaSys}

