% !TeX spellcheck = pl_PL
\newpage\section{Implementacja} \label{sec:implementacja}
% Dokumentacja programistyczna
% Odpowiedź na pytanie: Jak system zbudowano?
\subsection[Aplikacja mobilna]{Aplikacja mobilna [Damian Filipowicz]}
	\subsubsection{Przechowywanie danych}
	\subsubsection{Implementacja graficzna}
	\subsubsection{Walidacja danych wprowadzanych przez użytkownika}

\newpage
\subsection{Serwer}
	\subsubsection[Aplikacja serwerowa]{Aplikacja serwerowa [Maciej Marciniak]}
	\subsubsection[Strona internetowa]{Strona internetowa [Damian Filipowicz]}
\newpage
\subsection[Urządzenie sterujące]{Urządzenie sterujące [Maciej Marciniak]}

\newpage
\subsection[Moduł zliczania osób]{Moduł zliczania osób [Maciej Marciniak]}

\newpage
\subsection{Wnioski}

%rzyład: wzorzec arichtektoniczny MWK (ang. \textit{Model-Viewer-Controller}) \cite{mvc2017}.
%Zobacz listing \ref{lst:kod1}.

%\begin{lstlisting}[frame=single,captionpos=b,caption={Zawartość pliku \texttt{gvtopng.bat}},label={lst:kod1},basicstyle=\ttfamily]
% "c:\Program Files (x86)\Graphviz2.30\bin\dot.exe" ^
% -Tpng ucased.gv > ucased.png 
%\end{lstlisting}

%Korzystając z systemu Graphviz można wygenerować diagram UML (zob. rysunek \ref{fig:ucased}). Podpis umieszcza się pod rysunkiem.

 
