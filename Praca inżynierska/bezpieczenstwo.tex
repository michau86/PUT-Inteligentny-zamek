% !TeX spellcheck = pl_PL
\newpage
\section{Bezpieczeństwo systemu \textsl{\NazwaSys}} \label{sec:bezpieczenstwo}


\subsection{Techniki kryptograficzne [\StudentB]}
W systemie \textsl{\NazwaSys} zaimplementowano szereg funkcji kryptograficznych. Komunikacja pomiędzy modułami wymienionymi w punkcie \ref{Komunikacja serwer}, odbywa się przy pomocy Web API. Sama transmisja danych oparta jest o~protokół Https. Po stronie serwera przy pomocy paczek ''pySSl'' oraz ''Werkzug'' została zaimplementowana funkcja SSL dla wszystkich API włącznie z stroną internetową. Z racji braku posiadania własnego certyfikatu SSL w~projekcie tym został wygenerowany przy pomocy paczki ''Werkzug'' przykładowy certyfikat. Z~racji tego że ten urząd nie jest rozpoznawalny wyświetlają się komunikaty na~stronie internetowej o niebezpieczeństwie w postaci niezaufanego certyfikatu. W~aplikacji mobilnej na~czas korzystania z niego została zaimplementowana opcja zezwalająca na używanie certyfikatów z niezaufanego źródła. Podczas wdrażania projektu w prawdziwych warunkach zalecamy wykorzystanie prawdziwego certyfikatu.

Funkcje kryptograficzne użyte w aplikacji mobilnej należy rozdzielić na~dwie sekcje. Pierwszą z~nich są funkcje używane podczas przechowywania danych użytkownika. Wszystkie newralgiczne informacje są szyfrowane w celu polepszenia poufności w systemie. Dane przechowywane w pamięci telefonu szyfrowane są funkcją kryptograficzną AES w trybie blokowym CBC. W przypadku aplikacji mobilnej korzystaliśmy z gotowej implementacji zawartej w bibliotece ''crypto'' przeznaczonej dla języka Java. Dane szyfrowane są trzema rodzajami haseł i~tak hasło użytkownika przechowywane jest w pamięci telefonu (SharedPreferences) w postaci zaszyfrowanego tekstu, który jest szyfrowany hasłem zaszytym w~implementacje aplikacji. Klucz szyfrujący oraz token szyfrowane są za pomocą hasła użytkownika. Podczas eksportu klucza szyfrującego wraz z~certyfikatem klucza szyfrującego, które łączone są w jeden plik   występuje dodatkowe szyfrowanie hasłem które użytkownik wpisze. Zapewnia to zwiększoną poufność oraz poprawia bezpieczeństwo podczas przenoszenia pliku.                

Drugą sekcją są funkcje kryptograficzne użyte podczas generowania klucza szyfrującego oraz używania go do podpisów. Algorytmem użytym do tego jest RSA. Sygnatura wykorzystywana w tym algorytmie to SHA-256. Do implementacji tych funkcji skorzystaliśmy z gotowej biblioteki ''security''. Klucze te są podstawą kryptograficzną naszego projektu oraz implementacji PKI. Generowane są one po stronie aplikacji mobilnej i z pary kluczy prywatny publiczny tylko publiczny jest przesyłany do serwera, i dzięki temu oraz polityce przechowywania tego klucza uniemożliwiamy przejęcie tego klucza przez osoby niepowołane.

Aplikacja zarządzająca zamkiem korzysta z funkcji kryptograficznych takich jak RSA oraz SHA-256. Wykorzystuje ona je w procesie weryfikacji certyfikatu użytkownik wysłanego z urządzenia mobilnego. Do implementacji tych funkcji wykorzystano gotową bibliotekę ''PyCrypto''

\subsection{Podatności systemu (OWASP Top 10) [\StudentA]}

\subsection{Inne zagrożenia występujące w systemie [\StudentB]}

\subsection{Możliwości poprawy bezpieczeństwa systemu [\StudentA]}

\subsection{Wnioski}
 
