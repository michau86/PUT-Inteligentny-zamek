% 
\newpage\section{Zarys idei systemu \textsl{\NazwaSys}}\label{sec:ideasystemu}
\subsection{Schemat ideowy systemu \textsl{\NazwaSys}}

\subsection{Opis składowych systemu}
Nasz system składa się z 5 elementów.

Pierwszym z nich jest Urządzenei sterujące  w którego skład wchodzą  Raspberry
Pi 3 oraz serwomechanizm/zamek elektroniczny, jest weryfikacja klucza cyfrowego
przesyłanego przez urzadzenie mobilne oraz otwieranie zamka przy pozytywnym
wyniku weryfikacji.
Oprogramowanie mikrokomputera obejmuje system Linux raspbian-jessie oraz
szereg podprogramów napisanych w jezyku Python. Skrypty programów łacza sie
do serwera w celu pobrania informacji o poprawnosci i dacie waznosci certyfikatu
dostepu. Jesli dane beda poprawne to zostaje wysterowany serwomechanizm (lub
wysłany impuls do elektrozamka), który otwiera zamek, w przeciwnym przypadku
uzytkownik zostanie poinformowany o odmowie dostepu, a nieudana próba dostania
sie do systemu zarejestrowana zostanie w bazie danych wraz z danymi własciciela
klucza.

Drugim elementem jest 
Aplikacja mobilna napisana na platforme Android która ma na celu
przechowywanie w pamieci smartfona kluczy cyfrowych uzytkownika oraz mozliwosc
interakcji uzytkownika z systemem.

Kolejnym z elementów jest serwer wraz z stroną internetową.
Rola serwera w tym systemie jest przechowywanie danych dostepowych w
bazie danych MySQL oraz wykonywanie operacji zleconych przez administratora bądż użytkownika systemu. Dodatkowo serwer obsługuje strone internetową która wyświetla na bieżąco historię uzycia zamków w systemie.

Przedostatnim elementem składowym systemu jest baza danych która przechowuje wszystkie kluczwoem informacje systemu oraz udostępnia je serwerowi.

Ostatnim z skłądowych systemu jest oprogramowanie zliczajaće ilość sobó w danym pomieszceniu wraz z kamerą której zadnaiem jest obliczanie informacji o aktualnyej liczbie osBó w danym pomieszczeniu. 


\subsection{Podmioty systemu} 
W pracy inżynierskiej można wyodrębnić następujące podmioty:
	\begin{itemize}
	\item \textbf{Użytkownik niezalogowany} --- jest to użytkownik któy posiada aplikacje mobilną na swoim urządzeniu lecz nie wykonał procesu logowania,
	\item \textbf{Użytkownik niezarejestrowany} --- jest to użytkownik któy wysłał prośbę o zarejestrowanie lecz nie jest ona jeszcze zatwierdzona,
	\item \textbf{Użytkownik zalogowany} --- jest to uzytkownik któy przeszedł poprawnie proces logowania. Posiada on ograniczoną funkcjonalność aplikacji
	\item \textbf{Administrator} --- jest to użytkownik zalogowany który posiada uprawnienia administratora co wiaże się z pełnym dostępem do funkcji aplikacji mobilnej,
	\item \textbf{Serwer} --- jest to oprogramowanie zarządzajaće całym systemem,
	\item \textbf{Urządzenie sterujące} --- jest to oprogramowanie zarządzajaće dostępem fizycznym do pomieszczeń,
	\item \textbf{Oprogramowanie zliczające} --- jest to oprogramowanie zwracajaće w czasie rzeczywistym ilość osóB przebywających w danym pomieszczeniu,
\end{itemize}