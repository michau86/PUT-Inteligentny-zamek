% !TeX spellcheck = pl_PL
\documentclass[twoside]{article}
\usepackage[polish]{babel}
\usepackage[T1]{fontenc}
\usepackage[utf8]{inputenc}
\usepackage{polski}   
\usepackage{titlesec}
\usepackage{mdwlist}
\usepackage{enumitem}
\usepackage{indentfirst}
\usepackage{graphicx}   % rysunki
\usepackage{listings}   % kody programów
\usepackage{longtable}
\usepackage{array}
\usepackage{chngcntr}
\usepackage{pdflscape}
\usepackage{hyperref} 
\usepackage{color}
\usepackage{lastpage}

\usepackage[dvipsnames]{xcolor}

\definecolor{pblue}{rgb}{0.13,0.13,1}
\definecolor{pgreen}{rgb}{0,0.5,0}
\definecolor{pred}{rgb}{0.9,0,0}
\definecolor{pgrey}{rgb}{0.46,0.45,0.48}

\lstdefinelanguage{Kotlin}{
	keywords={package, as, typealias, this, super, val, var, fun, for, null, true, false, is, in, throw, return, break, continue, object, if, try, else, while, do, when, yield, typeof, yield, typeof, class, interface, enum, object, override, public, private, get, set, import, abstract, },
	keywordstyle=\color{NavyBlue}\bfseries,
	ndkeywords={@Deprecated, Iterable, Int, Integer, Float, Double, String, Runnable, dynamic},
	ndkeywordstyle=\color{BurntOrange}\bfseries,
	emph={println, return@, forEach,},
	emphstyle={\color{OrangeRed}},
	identifierstyle=\color{black},
	sensitive=true,
	commentstyle=\color{gray}\ttfamily,
	comment=[l]{//},
	morecomment=[s]{/*}{*/},
	stringstyle=\color{ForestGreen}\ttfamily,
	morestring=[b]",
	morestring=[s]{"""*}{*"""},
}


\lstset{language=Java,
	showspaces=false,
	showtabs=false,
	breaklines=true,
	showstringspaces=false,
	breakatwhitespace=true,
	commentstyle=\color{pgreen},
	keywordstyle=\color{pblue},
	stringstyle=\color{pred},
	basicstyle=\ttfamily,
	moredelim=[il][\textcolor{pgrey}]{$$},
	moredelim=[is][\textcolor{pgrey}]{\%\%}{\%\%}
}

\def\TytulPolski    {Projekt i wykonanie systemu kontroli ruchu i zarządzania dostępem do pomieszczeń}
\def\TytulAngielski {Design and implementation of movement control and access to spaces managment system }
 
% Nazwa systemu informatycznego
\def\NazwaSys {\textit{Inteligentny zamek}}

\def\Promotor    {dr inż. Ewa Idzikowska}
\def\StudentA     {Maciej Marciniak}
\def\AlbumA       {121996}
\def\StudentB     {Damian Filipowicz}
\def\AlbumB       {122002}
\def\Kierunek {Informatyka}
\def\Specjalnosc {Bezpieczeństwo systemów informatycznych} 
\def\PoziomStudiow {I stopnia}
\def\FormaStudiow {stacjonarne}
\AtBeginDocument{
    \renewcommand*{\tablename}{Tabela}
    \renewcommand*{\figurename}{Rys.} 
}
\setcounter{secnumdepth}{4}

\titleformat{\paragraph}
{\normalfont\normalsize\bfseries}{\theparagraph}{1em}{}
\titlespacing*{\paragraph}
{0pt}{3.25ex plus 1ex minus .2ex}{1.5ex plus .2ex}
\setlength{\parindent}{25pt}
\setlength{\parskip}{5pt}
\frenchspacing
\newcommand{\linia}{\rule{\linewidth}{0.4mm}}
\newcommand{\tablinia}{\newline \linia \newline}   
\counterwithin{figure}{section}
\counterwithin{table}{section}

\hypersetup{
	colorlinks=true,
	linkcolor=black,
	filecolor=black,      
	urlcolor=black,
	citecolor=black,
}

\lstset{literate=%
    {ż}{{\.z}}1
    {ą}{{\c{a}}}1
    {ę}{{\c{e}}}1
		{ó}{{\'o}}1
		{ć}{{\'c}}1
		{ś}{{\'s}}1
}

\definecolor{pblue}{rgb}{0.13,0.13,1}
\definecolor{pgreen}{rgb}{0,0.5,0}
\definecolor{pred}{rgb}{0.9,0,0}
\definecolor{pgrey}{rgb}{0.46,0.45,0.48}

\lstdefinelanguage{Kotlin}{
	keywords={package, as, typealias, this, super, val, var, fun, for, null, true, false, is, in, throw, return, break, continue, object, if, try, else, while, do, when, yield, typeof, yield, typeof, class, interface, enum, object, override, public, private, get, set, import, abstract, },
	keywordstyle=\color{NavyBlue}\bfseries,
	ndkeywords={@Deprecated, Iterable, Int, Integer, Float, Double, String, Runnable, dynamic},
	ndkeywordstyle=\color{BurntOrange}\bfseries,
	emph={println, return@, forEach,},
	emphstyle={\color{OrangeRed}},
	identifierstyle=\color{black},
	sensitive=true,
	commentstyle=\color{gray}\ttfamily,
	comment=[l]{//},
	morecomment=[s]{/*}{*/},
	stringstyle=\color{ForestGreen}\ttfamily,
	morestring=[b]",
	morestring=[s]{"""*}{*"""},
}

\lstset{
language=Python,
basicstyle=\ttm,
otherkeywords={self},             % Add keywords here
keywordstyle=\ttb\color{deepblue},
emph={MyClass,__init__},          % Custom highlighting
emphstyle=\ttb\color{deepred},    % Custom highlighting style
stringstyle=\color{deepgreen},
frame=tb,                         % Any extra options here
showstringspaces=false            % 
}

\lstset{language=Java,
	showspaces=false,
	showtabs=false,
	breaklines=true,
	showstringspaces=false,
	breakatwhitespace=true,
	commentstyle=\color{pgreen},
	keywordstyle=\color{pblue},
	stringstyle=\color{pred},
	basicstyle=\ttfamily,
	moredelim=[il][\textcolor{pgrey}]{$$},
	moredelim=[is][\textcolor{pgrey}]{\%\%}{\%\%}
}
     
\lstset{
	numbers=left, 
  	numberstyle=\footnotesize, 
   	numbersep=3pt, 
   	frame = single, 
   	language=Java, 
   	framexleftmargin=12pt
}  



\title{\TytulPolski}
\author{\StudentA}

\begin{document}
% strona tytułowa: Politechnika Poznańska  Wydział Elektryczny  Instytut Automatyki i Inżynierii Informatycznej
% Imie Nazwisko
% Praca dyplomowa inżynierska
% Tutuł pracyy 
% promotor: dr inż. Imię Nazwisko
% Poznań, 2017
% \maketitle
\thispagestyle{empty}
\begin{center}
Politechnika Poznańska\\
Wydział Elektryczny\\  
Instytut Automatyki i Inżynierii Informatycznej\\
 % \vspace{5mm}
\begin{figure}[ht!]
%\centering
\centering
\includegraphics[width=50mm]{pplogo.png}
\end{figure}
  \vspace{5mm}
\large{\StudentA}\\
\large{\StudentB}\\
  \vspace{10mm}
\large{\TytulPolski}\\
  \vspace{10mm}
\large{Praca dyplomowa inżynierska}\\
\end{center}
\vspace{40mm}
\begin{flushright}
promotor:\\
\Promotor
\end{flushright}

\vspace{10mm}
\begin{center}
Poznań, 2018
\end{center}
 
        % strona tytułowa
% strona z karta tematu z dziekanatu (zrobic ksero)
\newpage
\textit{karta pracy umieszczona tylko informacyjnie}

''Karta Pracy Damian Filipowicz''
\newpage
\textit{karta pracy umieszczona tylko informacyjnie}

\includegraphics[width=13.2cm]{Karta_pracy_Maciej}
\null 

        % karta tematu
% !TeX spellcheck = pl_PL
\newpage
\thispagestyle{empty}
\begin{center}
Poznan University of Technology\\
Faculty of Electrical Engineering\\
Institute of Control, Robotics and Information Engineering\\
  \vspace{15mm}
\huge{\TytulAngielski} \\
\large{by}\\
  \vspace{5mm}
\large{\StudentA}\\
\large{\StudentB}\\
  \vspace{15mm}

\normalsize\textbf{Abstract} \\
{Thesis work presents \TytulAngielski. The basic assumption is to supervise the opening and closing of doors in the area of a large building where there are many entrances. The system stores data about users and their access rights, when and where they can enter. The presented effects of the system are divided into a mobile application and a website. The rest of the system (control device, server) provides a fully logical role.} 

\end{center}

\begin{center}
 \textbf{Streszczenie} \\
 {Praca dyplomowa przedstawia \TytulPolski. Podstawowym założeniem jest nadzorowanie otwierania i zamykania drzwi w obrębię dużego budynku, gdzie znajduje się wiele wejść. System przechowuje dane na temat użytkowników oraz ich uprawnień dostępu, kiedy i gdzie mogą wchodzić. Przedstawiane efekty działania systemu podzielone są na aplikację mobilną oraz stronę internetową. Pozostała część systemu (urządzenia sterujące, serwer) zapewnią w pełni logiczną rolę.} 
\end{center}

 
        % strona w j. angielskim i streszczenie
\newpage\tableofcontents     % Spis tresci
\newpage\section{Wstęp}\label{sec:wstep}
 
\subsection{Cel i zakres pracy}
% Odpowiedź na pytanie: Do czego system służy?
% Przedswić zadania szczegółowe z karty tematu 

\subsection{Plan pracy}
% Spis tresci napisany słownie

\subsection{Harmonogram pracy}
% wykres Gantta

\subsection{Metodyka pracy grupowej}

 
          % wstep
% 
\newpage\section{Opis dziedziny przedmiotowej pracy}\label{sec:dziedzina}
\subsection{Pojęcia i definicje}
W dokumencie tym posługiwać się będziemy następującymi pojęciami:


	\begin{itemize}
	\item \textbf{Klucz dostępowy} --- jest to klucz publiczny z pary kluczy prywartny publiczny. Używany jest on do odszyfrowania wiadomości wysłanej z aplikacji mobilnej do urządzenia sterujacego,
	
	\item \textbf{Klucz szyfrujący} 
	--- jest to klucz prywatny wygenerowany podczas tworzenia pary kluczy publiczny prywatny. Używane jest on do szyfrowania wiadomości wysyłanej z aplikacji mobilnej do urządzenia sterującego,
	
    \item \textbf{Para kluczy szyfrujących}
    --- jest to para kluczy (prywatny oraz publiczny) generowanych podczas rejestracji oraz wymiany klucza dostepowego,
	
	\item \textbf{Inteligentny zamek}
	--- system obsługujący otwieranie elektrozamka bądz serwomechanizmu,
    \item \textbf{Konto}
	--- reprezentacja użytkownika w systemie za pomocą takich danych jak login hasło, imie, nazwisko.
	\item \textbf{Administrator}
	---jest to fizyczna osoba posiadająća dla swojego konta uprawnienia administratora co wiąże się z pełnym dostępem do systemu.					
\end{itemize}



\newpage
\subsection{Stan wiedzy}
Przed przystąpieniem do projektu zrobiliśmy porównanie systemów zbliżonych do naszego który na dany moment istniały. I tak doszliśmy do wniosku że wszystkie systemy inteligentnych zamków wykonane przez firmy takie jak Gerda Lock czy Danalock zostały wykonane typowo dla użytku domowego a nie tak jak nasz projekt inżynierski który jest przeznaczony do zarządzania w budynkach o wielu pomieszczeniach z różnym stopniem dostępu. Opis wraz z porównaniem poszcególnych systemów znajduje się w tabelach poniżej.


	Tabela \ref{tab:porownanie1} zawiera porównanie firm pod względem otwierania zamka
	\begin{longtable}[!ht]{|p{4cm}|p{1,5cm}|p{1,5cm}|p{1,5cm}|p{1,5cm}|} 
		\caption{Tabela porównania otwierania zamków}
		\label{tab:porownanie1}\\
		\hline	
		 & NOKI & August & DanaLock & Gerda Lock  \\	\hline
		 zarządzanie wieloma zamkami z jednej aplikacji
		 & brak & tak & brak & brak \\	\hline
		otwieranie zamka przy pomocy strony WWW
		& brak & brak & tak & brak \\	\hline
		inne sposoby otwarcia zamka niż aplikacja
		& brak informacji& brak informacji & brak & tak \\	\hline
		automatyczne zamykanie zamka
		& brak informacji& tak & tak & tak   \\	\hline
		tryb otwierania zamka automatycznie
		& tak& brak & tak & tak \\	\hline	
		tryb otwierania zamka po zezwoleniu przyciskiem
		& brak & tak & tak & tak \\		\hline
	\end{longtable}


 
 	Tabela \ref{tab:porownanie2} zawiera porównanie firm pod względem zasilania i montażu
 \begin{longtable}[!ht]{|p{4cm}|p{1,5cm}|p{1,5cm}|p{1,5cm}|p{1,5cm}|} 
 	\caption{Tabela porównania zasialania i montażu}
 	\label{tab:porownanie2}\\
 	\hline	
 	& NOKI & August & DanaLock & Gerda Lock  \\	\hline
 	zasilanie zewnętrzne (z sieci)	
 	& brak & brak & brak & brak \\	\hline
	 zasilanie bateryjne (podstawowe/ awaryjne)	
	 & podstawowe & podstawowe & podstawowe & podstawowoe \\	\hline
 	sposób montażu	
 	& nakłądka na zamek & nakłądka na zamek & nakłądka na zamek & nakłądka na zamek \\	\hline
 \end{longtable}
 



Tabela \ref{tab:porownanie3} zawiera porównanie firm pod względem dziennika zdarzeń oraz powiadomień
\begin{longtable}[!ht]{|p{4cm}|p{1,5cm}|p{1,5cm}|p{1,5cm}|p{1,5cm}|} 
	\caption{Tabela porównania zasialania i montażu}
	\label{tab:porownanie3}\\
	\hline	
	& NOKI & August & DanaLock & Gerda Lock  \\	\hline
	podgląd kto otworzył	
	& brak informacji & brak & brak & tak \\	\hline
	
	
	powiadomienie o otwarciu drzwi (ogólnie i przez daną osobę)
	& brak & brak & brak & tak \\	\hline
	
	
	powiadomienie o nieautoryzowanych próbach otwarcia
	& tak & brak & brak & tak \\	\hline
\end{longtable}

\newpage
\subsection{Stan pracy wykonany w ramach zajęć \newline przedmiotowych} 
W ramach zajęć projektowych oraz laboratoryjnych o nazwe Projekt Zespołowy prowadzonych z mgr. Michałem Apolinarskim oraz dr Ewą Idzikowską zostały wykonane następujace fragmenty systemu:
	
	Aplikacja mobilna została wykonana dla wersji andorida minimum 4.4 KitKat w stopniu umożliwiającym takie funkcjonalnośći jak:
	\begin{itemize}
		\item Logowanie
		\item Rejestracja
		\item Rejestracja wraz z tworzeniem pary kluczy dostępowych publiczny prywatny
		\item Generowanie nowego certyfikatu
		\item Pobieranie certyfikatów z serwera
		\item Zarządznanie certyfikatami użytkownika
		\item Zarządzanie prośbami o rejestracje
		\item Wnioskowanie o certyfikat nowy
	\end{itemize}
		Dodatkowo zostało napisane api do obsługi połączenia bluetooth oraz w każdym widoku któy korzystał z połaczenia z serwerem były napisane fragmenty kodu. Funkcje te oraz kod zostały napisane bez uwzględnienia wzorców architektoniczncych (wszystko co dotyczyło danego widoku było w jednej klasie), posiadały szereg błędów powodujaćych niestabilne działąnie systemu oraz posiadały metody z systemu android które były określane przez środowisko android stuido jako "deprecated" co mogło przy nowszych wersjach androida powodować wadliwe działanie systemu. Z racji pisania pod wersje systemu android 4.4 wygląd różni się od tego który został zaimplementowany w pracy inżynierskiej.
		 
	
	
   Aplikacja serwerowa została wykonanan w stopniu umożliwiającym podstawowoe funkcje takie jak:
   		\begin{itemize}
   		\item Logowanie
   		\item Rejestracja
   		\item Zapisywanie nowego certyfikatu
   		\item udostępnianie certyfikatóW
   		\item pobieranie list certyfikatów, próśb o certyfikaty oraz rejestracji
   	\end{itemize}
	Wszystkie te funkcje zwracały odpowiednio albo pożądane dane albo wartość 'Invalid' co powodowało wyświetlania błędnych komunikaótw uzytkownikowi z racji nierozróżniania braku dostępu od błędnie wykonanego skyptu.
	
	Urządzenie sterujące zamkiem zostało napisane w języku python i pozwalało na odbieranie danych z aplikacji mobilnej oraz posiadało funckje odpowiedzialną za otwieranie zamka.
 

	W ramach przedmiotu ochrona danych zostały zaimplementowane w systemie fragmenty PKI takie jak:
	   \begin{itemize}
	   	\item Certyfikat klucza dostępowego
	   	\item  generowanei nowego Certyfikatu uzytkownika
	   	\item blokowanie uzytkownika systemu
	   \end{itemize}
	 Funkcje te zostały napisane zarówno po stronie aplikacji mobilnej jak i aplikacji serwerowej. Ponadto po stronie androida został opracowany sposób przechowywania klucza prywatnego w formie zaszyfrowanego pliku hasłem uzytkownika.
      % wprowadznie do tematu pracy 
% 
\newpage\section{Zarys idei systemu \NazwaSys}\label{sec:ideasystemu}
\subsection{Schemat ideowy systemu \NazwaSys}
\subsection{Opis składowych systemu}
\subsection{Podmioty systemu} 	   % zarys idei systemu
\newpage\section{Wybór technologii informatycznych} \label{sec:technologie}
\subsection{Urządzenie sterujące}
\subsection{Aplikacja serwera}
\subsection{Aplikacja mobilna}
\subsection{Moduł zliczania osób}
\subsection{System kontroli wersji}
\subsection{Prowadzenie dokumentacji}
 
    % opis technologii informatycznych wykorzystanych w pracy
% !TeX spellcheck = pl_PL
% 
\newpage\section{Projekt systemu \textsl{\NazwaSys}}\label{sec:projekt}
\subsection{Diagramy UML}

	\subsubsection{Diagramy przypadków użycia}
	
		\paragraph{Aplikacja mobilna}
		\paragraph{Aplikacja serwera}
		\paragraph{Urządzenie sterujące}
		\paragraph{Moduł zliczania osób}
		
	\subsubsection{Diagramy sekwencji systemu}
	
		\paragraph{Aplikacja mobilna}
		\paragraph{Aplikacja serwera}
		\paragraph{Urządzenie sterujące}
		\paragraph{Moduł zliczania osób}
		
	\subsubsection{Projekt bazy danych} 
	
	\subsubsection{Diagramy klas} 
	
		\paragraph{Aplikacja mobilna}
		Aplikacja mobilna składa się z szeregu klas napisanych w 2 językach: kotlin oraz java. Ponadto klasy te 
		zostały podzielone na 5 kategorii takich jak:
		\begin{itemize*}
			\item API 
			(rysunek \ref{Diagram klas dla paczki api}) 
			--  które przechowywuje klasy odpowiedzialne za funkcje wykorzystywane w wielu miejscach systemu ,
			\item Navigation 
			(rysunek \ref{Diagram klas dla paczki navigations}) 
			-- są to klasy odpowiedzialne za generowanie nawigacji w apikacji mobilnej,
			\item Adapters
			(rysunek \ref{Diagram klas dla paczki adapters}) 
			 -- w którym są przechowywane klasy adapter wykorzystywane w systemie do wyświetlania dnaych,
		\end{itemize*}
		Oprócz tych wymienionych wyżej są jeszce 3 kategorie implementujące wzorzech architektoniczny Model-View-Presenter i są to odpowiednio:
			\begin{itemize*}
				\item Model
				(rysunek \ref{Diagram klas dla paczki models}) 
				 -- przechowywujący klasy modele odpoweidzialne za przechowywanie danych,   
				\item view 
				(rysunek \ref{Diagram klas dla paczki views}) 
				-- przechowywująćy klasy widoków odpowiedzialne za generowanie widoków w aplikacji, 
				\item presenter
				(rysunek \ref{Diagram klas dla paczki presenters}) 
				 -- przechowywujaće klasy presenter odpowiedzialne za interakcje pomięczy modalami oraz widokami.
			\end{itemize*}
		
	
		\begin{figure}[ht!]
			\centering
			\includegraphics[width=12.5cm,height=6cm,keepaspectratio]{Obrazy/AM_DK_ALL}
			\caption{Schemat ogólny diagramu klas dla Aplikacji Mobilnej}
			\label{Schemat ogólny diagramu klas dla Aplikacji mobilnej}
		\end{figure}
	

		\begin{figure}[ht!]
		\centering
		\includegraphics[width=12.5cm,height=10cm,keepaspectratio]{Obrazy/AM_DK_adapter}
		\caption{Diagram klas dla paczki adapters}
		\label{Diagram klas dla paczki adapters}
	\end{figure}


	
		\begin{figure}[ht!]
		\centering
		\includegraphics[width=12.5cm,height=10cm,keepaspectratio]{Obrazy/AM_DK_navigation}
		\caption{Diagram klas dla paczki navigations}
		\label{Diagram klas dla paczki navigations}
	\end{figure}

	
	
	
	
	\begin{figure}[ht!]
		\centering
		\includegraphics[width=12.5cm,height=16cm,keepaspectratio]{Obrazy/AM_DK_api}
		\caption{Diagram klas dla paczki api}
		\label{Diagram klas dla paczki api}
	\end{figure}

	

	\begin{figure}[ht!]
		\centering
		\includegraphics[width=12.5cm,height=16cm,keepaspectratio]{Obrazy/AM_DK_model}
		\caption{Diagram klas dla paczki models}
		\label{Diagram klas dla paczki models}
	\end{figure}

	
\begin{figure}[ht!]
	\centering
	\includegraphics[width=12.5cm,height=12cm,keepaspectratio]{Obrazy/AM_DK_view}
	\caption{Diagram klas dla paczki views}
	\label{Diagram klas dla paczki views}
\end{figure}

		
			Ostatni diagram dotycząćy aplikacji mobilnej przedstawia rozwinięcie paczk Presenter wraz z połączeniami z innymi paczkami
		\begin{figure}[ht!]
			\centering
			\includegraphics[width=12.5cm,height=14cm,keepaspectratio]{Obrazy/AM_DK_presenter}
			\caption{Diagram klas dla paczki presenters}
			\label{Diagram klas dla paczki presenters}
		\end{figure}
		
		\newpage
		\paragraph{Aplikacja serwera}
		\paragraph{Urządzenie sterujące}
		\paragraph{Moduł zliczania osób}

\newpage		
\subsection{Uproszczony schemat elektryczny systemu}

\newpage
\subsection{Komunikacja modułów systemu z aplikacją  serwera}

	\subsubsection{Komunikaty HTTPRequest pomiędzy aplikacją mobilną, \newline a serwerem}
	\subsubsection{Komunikaty HTTPRequest pomiędzy urządzeniem sterującym, a serwerem}
	
\newpage
\subsection{Protokoły komunikacji pomiędzy urządzeniem \newline sterującym i aplikacją mobilną}

\newpage
\subsection{Interfejs graficzny systemu}

	\subsubsection{Widoki aplikacji mobilnej}
		\section*{Panel logowania użytkownika}
	Widok umożliwia zalogowanie się użytkownika do systemu poprzez podanie loginu, hasła oraz adres IP serwera w odpowiednie pola, a następnie kliknięcie w przycisk “ZALOGUJ SIĘ”. Samo pole hasła jest maskowane. Jeśli nie posiada się konta, można je utworzyć poprzez przycisk “ZAREJESTRUJ SIĘ”. (Rysunek \ref{rys:panel_logowania_pionowo})
	
	\begin{figure}[ht!]
			\centering
			\includegraphics[width=12.5cm,height=8cm,keepaspectratio]
			{Obrazy/logowanie_uzytkownika_pionowo}
			\caption{Panel logowania użytkownika}
			\label{rys:panel_logowania_pionowo}
	\end{figure}

	
	\section*{Panel rejestracji użytkownika}
	Panel rejestracji służy do utworzenie nowego użytkownika poprzez podanie loginu, hasła, imienia i nazwiska użytkownika oraz adresu IP serwera do którego chcemy się zarejestrować. Pole z hasłem jest maskowane wraz z możliwośćią odkrywania hasłą przy pomocy ikonki oko Po upewnieniu się, że wszystkie dane są poprawne, aby zakończyć proces rejestracji, klikamy przycisk “ZAREJESTRUJ”. (Rysunek \ref{rys:panel_rejestracji_pionowo})
	
	\begin{figure}[ht!]
		\centering
		\includegraphics[width=12.5cm,height=8cm,keepaspectratio]
			{Obrazy/rejestracja_uzytkownika_pionowo}
			\caption{Panel logowania użytkownika }
			\label{rys:panel_rejestracji_pionowo}
		
	\end{figure}
	
	
	\section*{Panel listy zamków}
	Widok listy dostępnych zamków przedstawia listę nazw zamków do jakich dany użytkownik ma dostęp. Ułatwieniem jest możliwość sortowania wyników i wyszukiwanie po nazwach. Kliknięcie w nazwę zamka powoduje otwarcie zamka. Zmiana koloru ikon zamków sygnalizować ma status zamka.Opisy znaczeń poszcególnych koloróW oraz symboli opisane są w rozdziale symbolika oraz kolory. (Rysunek \ref{rys:panel_listy_dostepnych_zamkow_pionowo})
	
	\begin{figure}[ht!]
			\centering
		\includegraphics[width=12.5cm,height=8cm,keepaspectratio]
			{Obrazy/lista_dostepnych_zamkow_pionowo}
			\caption{Lista dostępnych zamków}
			\label{rys:panel_listy_dostepnych_zamkow_pionowo}
	
		
	\end{figure}
	
	
	\section*{Panel boczny}
	Panel boczny pozwala na szybkie przełączanie pomiędzy widokami. Chowany jest po lewej stronie ekranu. Umożliwia przechodzenie odpowiednio do listy zamków, zarządzania certyfikatami, panelu administracyjnego oraz ustawień. Ostatnia pozycja powoduje wylogowanie z aplikacji. Pnanel ten w zależnośći od uprawnień użytkownika możę posiadać lub nie pole z panelem administraotra (Rysunek \ref{rys:panel_boczny_pionowo} i \ref{rys:panel_boczny_pionowo2})
	
	\begin{figure}[ht!]
		\begin{minipage}{0.5\textwidth}
			\includegraphics[width=\textwidth]
			{Obrazy/panel_boczny_pionowo}
			\caption{Panel boczny z uprawnieniami administratora}
			\label{rys:panel_boczny_pionowo}
		\end{minipage}
		\begin{minipage}{0.5\textwidth}
			\includegraphics[width=\textwidth]{Obrazy/panel_boczny_pionowo2}
			\caption{Panel boczny bez uprawnieni administratora}
			\label{rys:panel_boczny_pionowo2}
		\end{minipage}	
	\end{figure}

	
	\section*{Panel zarządzania certyfikatami}
	Panel zarządzania certyfikatami umożliwia wybór funkcji dodania certyfikatu. Kolejne pozycje to lista posiadanych certyfikatów oraz wysłanie wniosku o utworzenie nowego certyfikatu  (Rysunek \ref{rys:panel_zarządzania_certyfikatami_pionowo})
	
	\begin{figure}[ht!]
		\centering
		\includegraphics[width=12.5cm,height=8cm,keepaspectratio]
			{Obrazy/zarzadzaj_certyfikatami_pionowo}
			\caption{Panel zarządzania certyfikatami }
			\label{rys:panel_zarządzania_certyfikatami_pionowo}
		
	\end{figure}

	
	\section*{Panel listy certyfikatów}
	Panel listy certyfikatów, jest listą aktualnych certyfikatów należących do użytkownika. Kliknięcie w dany certyfikat przenosi do widoku szczegółowego związanego z operacjami na tym certyfikacie. (Rysunek \ref{rys:panel_listy_certyfikatów_pionowo} )
	
	\begin{figure}[ht!]
			\centering
	\includegraphics[width=12.5cm,height=8cm,keepaspectratio]
			{Obrazy/lista_certyfikatow_pionowo}
			\caption{Panel listy certyfikatów}
			\label{rys:panel_listy_certyfikatów_pionowo}
		
	\end{figure}

	
	\section*{Panel certyfikatu}
	Panel certyfikatu zawiera informacje o dacie wygaśnięcia, którego zamku dotyczy oraz w jakim czasie przyznaje dostęp. Na dole dostępne są dwa przyciski pozwalające usunąć certyfikat lub wysłać prośbę o przedłużenie ważności. W zależnośći od tego czy uzytkownik ma uprawnienia administratora przycisk przedłuż certyfikat albo wyśle zgłoszenie do serwera (dla użytkownika bez uprawnieni administratora) albo przeniesie do panelu generowania certyfikatu (dla użytkownika o uprawnieniach administratora) (Rysunek \ref{rys:panel_certyfikatu_pionowo})
	
	\begin{figure}[ht!]
		\centering
		\includegraphics[width=12.5cm,height=8cm,keepaspectratio]
			{Obrazy/certyfikat_pionowo}
			\caption{Panel certyfikatu }
			\label{rys:panel_certyfikatu_pionowo}
		
	\end{figure}
	
	\section*{Panel wnioskowania o certyfikat}
	Panel wnioskowania o certyfikat polega na wybraniu z listy wszystkich zamków, konkretnego do którego chcemy uzyskać dostęp i wysłać wniosek o przydzielenie dostępu. (Rysunek \ref{rys:panel_wnioskowania_o_certyfikat_pionowo})
	
	\begin{figure}[ht!]
		\centering
		\includegraphics[width=12.5cm,height=8cm,keepaspectratio]
			{Obrazy/wnioskuj_o_certyfikat_pionowo}
			\caption{Panel wnioskowania o certyfikat }
			\label{rys:panel_wnioskowania_o_certyfikat_pionowo}
	
	\end{figure}
	
	
	\section*{Panel administratora}
	W panelu administratora znajdują się 6 przycisków do administrowania systemem zamków:
	\begin{itemize*}
		\item ,,Historia użycia zamków''
		\item ,,Generowanie nowego certyfikatu'',
		\item ,,Zarządzanie certyfikatami użytkowników'',
		\item ,,Lista oczekujących użytkowników do zarejestrowania'',
		\item ,,Lista oczekujących certyfikatów do zaakceptowania'',
		\item ,,Zarządzanie kontami użytkowników''.
	\end{itemize*}
	
	Po kliknięciu każdego przycisku przechodzi się do nowego odpowiadającego widoku. (Rysunek \ref{rys:panel_administracyjny_pionowo})
	
	\begin{figure}[ht!]
			\centering
	\includegraphics[width=12.5cm,height=8cm,keepaspectratio]
			{Obrazy/panel_administracyjny_pionowo}
			\caption{Panel administratora}
			\label{rys:panel_administracyjny_pionowo}
	
	\end{figure}

	
	\section*{Panel historii użycia zamków}
	Panel historii użycia zamków składa się z rozwijanej listy filtorwania histori w której są elementy takie jak lista dostępnych zamkóW, lista dostępnych uzytkowników, data podług której następuje filtracja oraz chekbox do zaznaczania czy tylko były nieautoryzowane próby. By uzyskać daną filtrację należy nacisnać przycisk filtruj. Oprócz panelu do filtrowania znajduje się również sama historia gdzie wyświetlane jestrodzaj próby otwarcia, data oraz przez kogo była ta próba podjęta. (Rysunek \ref{rys:panel_historii_uzycia_zamka_pionowo} i \ref{rys:panel_historii_uzycia_zamka_pionowo2})
	
	\begin{figure}[ht!]
		\begin{minipage}{0.5\textwidth}
			\includegraphics[width=\textwidth]{Obrazy/historia_zamkow_pionowo}
			\caption{Panel historii użycia zamków (filtr)}
			\label{rys:panel_historii_uzycia_zamka_pionowo}
		\end{minipage}
		\begin{minipage}{0.5\textwidth}
			\includegraphics[width=\textwidth]{Obrazy/historia_zamkow_pionowo2}
			\caption{Panel historii użycia zamków (historia)}
			\label{rys:panel_historii_uzycia_zamka_pionowo2}	
		\end{minipage}
	\end{figure}
	\newpage
	
	\section*{Panel generowania nowego certyfikatu (administrator)}
	Panel generowanie nowego certyfikatu (administrator) służy do tworzenia nowych certyfikatów przez administratora. W pierwszych polach podaje się imię i nazwisko kogo dotyczy certyfikat.Następnie wybierane jest użytkownik (login) oraz zamek z rozwijanej listy. W dalszej części wybierane jest zakres dat w których certyfikat ma być ważny. Potem widać przycisk o nazwie "zakres obowiązywania certyfikatów" który przekierowywuje do widoku odpowiedzialnego za to w jakich godzinach dla danych dni tygodni certyfikat udziela dostępu. (Rysunek \ref{rys:panel_generowanie_nowego_klucza_gosc_admin_pioniowo}, \ref{rys:panel_generowanie_nowego_klucza_admin_pionowo2} i 
	\ref{rys:panel_wyboru_zakresu_certyfikatu})
	
	\begin{figure}[ht!]
		\vspace{-0.5cm}
		\begin{minipage}{0.5\textwidth}
			\includegraphics[width=\textwidth]{Obrazy/generowanie_nowego_klucza_gosc_admin_pioniowo}
			\caption{Panel generowania nowego klucza cz. 1 }
			\label{rys:panel_generowanie_nowego_klucza_gosc_admin_pioniowo}
		\end{minipage}
		\hspace{0.5cm}
		\begin{minipage}{0.5\textwidth}
			\includegraphics[width=\textwidth]{Obrazy/generowanie_nowego_klucza_admin_pionowo2}
			\caption{Panel generowania nowego klucza cz. 2}
			\label{rys:panel_generowanie_nowego_klucza_admin_pionowo2}	
		\end{minipage}
	\end{figure}
	\vspace{-0.5cm}
	\begin{figure}[ht!]
		\center
			\includegraphics[width=4.5cm]{Obrazy/generowanie_nowego_klucza_uzytkownik_zalogowany_admin_pioniowo}
			\caption{Panel generowania nowego klucza dla użytkownika }
			\label{rys:panel_generowanie_nowego_klucza_uzytkownik_zalogowany_admin_pioniowo}
	\end{figure}

	
	\section*{Panel zarządzania certyfikatami~(administrator)}
	Panel zarządzania certyfikatami użytkowników (administrator) jest widokiem tylko wszystkich aktywnych certyfikatów w systemie. Administrator klikając na pozycję przechodzi do panelu certyfikatu opisanego wyżej. Tam może usunąć dostęp lub go przedłużyć. Ułatwieniem jest możliwość wyboru typu sortowania. (Rysunek \ref{rys:panel_lista_certyfikatow_administrator_pionowo})
	
	\begin{figure}[ht!]
		\centering
	\includegraphics[width=12.5cm,height=8cm,keepaspectratio]
			{Obrazy/lista_certyfikatow_administrator_pionowo}
			\caption{Panel zarządzania certyfikatami (administrator) }
			\label{rys:panel_lista_certyfikatow_administrator_pionowo}
	
	\end{figure}

	
	\section*{Panel~listy~oczekujących~użytkowników~do~rejestracji}
	Panel listy oczekujących użytkowników jest listą wszystkich gości, którzy ubiegają się o zarejestrowanie. PO kliknięciu w odpowiednią pozycję pojawiają się dwie opcję: “AKCEPTUJ” lub “ODRZUĆ”.  (Rysunek \ref{rys:panel_lista_oczekujacych_uzytkownikow_pionowo} )
	
	\begin{figure}[ht!]
		\centering
	\includegraphics[width=12.5cm,height=8cm,keepaspectratio]
			{Obrazy/lista_oczekujacych_uzytkownikow_pionowo}
			\caption{Panel listy oczekujących użytkowników }
			\label{rys:panel_lista_oczekujacych_uzytkownikow_pionowo}
	
	\end{figure}

	
	\section*{Panel~listy~oczekujących~certyfikatów do~wygenerowania}
	Panel listy oczekujących certyfikatów jest listą wszystkich certyfikatów, które ubiegają się o akceptację administratora. Po kliknięciu w odpowiednią pozycję pojawiają się dwie opcję: “AKCEPTUJ” lub “ODRZUĆ”.  (Rysunek \ref{rys:panel_lista_oczekujacych_certyfikatow_pionowo} )
	
	\begin{figure}[ht!]
			\centering
			\includegraphics[width=12.5cm,height=8cm,keepaspectratio]
			{Obrazy/lista_oczekujacych_certyfikatow_pionowo}
			\caption{Panel listy oczekujących certyfikatów }
			\label{rys:panel_lista_oczekujacych_certyfikatow_pionowo}
		
	\end{figure}
	
	\section*{Panel zarządzania kontami użytkowników}
	Panel ten służy do zarządzania kontami użytkowników. Wyświetla on listę użytkownikóW systemu wraz z zonaczeniami czy jest on aktyny bądż zablokowany oraz czy ma ważny klucz szyfrujący (Rysunek \ref{rys:panel_Zarządzania_Kontami})
	
	\begin{figure}[ht!]
			\centering
			\includegraphics[width=12.5cm,height=8cm,keepaspectratio]
			{Obrazy/zarzadzanie_kontami}
			\caption{Panel zarządzania kontami użytkowników}
			\label{rys:rys:panel_Zarządzania_Kontami}
		
	\end{figure}
	
	\section*{Panel ustawień konta}
	W panelu ustawień użytkownik może zmienić hasło do swojego konta. Wymagane jest podanie starego hasła, a następnie nowego.Ponadto w panelu tym mamy podgląD certyfikatu szyfrujaćego wraz z możliwośćią wygenerowania nowego oraz zmiane adresu ip serwera  (Rysunek \ref{rys:panel_ustawienia_pionowo} i \ref{rys:panel_ustawienia_poziomo})
	
	\begin{figure}[ht!]
			\centering
			\includegraphics[width=12.5cm,height=8cm,keepaspectratio]
			{Obrazy/ustawienia_1}
			\caption{Panel ustawień konta (zmiana hasła)}
			\label{rys:panel_ustawienia_pionowo}
	\end{figure}
	\begin{figure}[ht!]
			\centering
		\includegraphics[width=12.5cm,height=8cm,keepaspectratio]
	{Obrazy/ustawienia_2}
	\caption{Panel ustawień konta (certyfikat szyfrująćy)}
	\label{rys:panel_ustawienia_pionowo}
\end{figure}
	\begin{figure}[ht!]
			\centering
		\includegraphics[width=12.5cm,height=8cm,keepaspectratio]
	{Obrazy/ustawienia_3}
	\caption{Panel ustawień konta (adres ip)}
	\label{rys:panel_ustawienia_pionowo}
\end{figure}
	\newpage
	
	
	
	
	\subsubsection{Widoki strony internetowej systemu}
	Strona internetowa posiada dwa widoki. Jeden jest to widok logowania  (Rysunek \ref{rys:strona_1} w któym administrator musi wpisać login oraz hasło. W drguim widoku mamy listę histori otwarcia zamków  (Rysunek \ref{rys:strona_2}) wraz z zaznaczeniem kolorystycznym która pokazuje czy była to próba autoryzowana.
	

\begin{figure}[ht!]
		\centering
	\includegraphics[width=12.5cm,height=10cm,keepaspectratio]
{Obrazy/strona_logowanie}
\caption{Strona logowania}
\label{rys:strona_1}
\end{figure}


\begin{figure}[ht!]
		\centering
	\includegraphics[width=12.5cm,height=10cm,keepaspectratio]
{Obrazy/strona_historia}
\caption{Strona z wyświetloną historią użycia zamkóW)}
\label{rys:strona_2}
\end{figure}
	
	\subsubsection{Komunikacja człowiek-interfejs}
	
		\paragraph{Komunikaty tekstowe}
		\paragraph{Symbolika ikon}
		\paragraph{Znaczenie kolorystyki}
		
	\subsubsection{Kolorystyka systemu}
	
\newpage
\subsection{Bezpieczeństwo systemu}
	\subsubsection{Projekt infrastruktury klucza publicznego (PKI)}
		\paragraph{Idea PKI}
		\paragraph{Urzedy certyfikujące}
		\paragraph{Klient systemu}
	\subsubsection{Poufność}
	\subsubsection{Dostępność}
	\subsubsection{Integralność}        % dokumentacja projektowa
% !TeX spellcheck = pl_PL
\newpage\section{Implementacja} \label{sec:implementacja}
% Dokumentacja programistyczna
% Odpowiedź na pytanie: Jak system zbudowano?
\subsection{Aplikacja mobilna}
	\subsubsection{Przechowywanie danych}
	\subsubsection{Graficzna implementacja}
	\subsubsection{Walidacja danych wprowadzanych przez użytkownika}

\newpage
\subsection{Aplikacja serwerowa}
	\subsubsection{Strona internetowa}

\newpage
\subsection{Urządzenie sterujące}

\newpage
\subsection{Moduł zliczania osób}

\newpage
\subsection{Wnioski}

%rzyład: wzorzec arichtektoniczny MWK (ang. \textit{Model-Viewer-Controller}) \cite{mvc2017}.
%Zobacz listing \ref{lst:kod1}.

%\begin{lstlisting}[frame=single,captionpos=b,caption={Zawartość pliku \texttt{gvtopng.bat}},label={lst:kod1},basicstyle=\ttfamily]
% "c:\Program Files (x86)\Graphviz2.30\bin\dot.exe" ^
% -Tpng ucased.gv > ucased.png 
%\end{lstlisting}

%Korzystając z systemu Graphviz można wygenerować diagram UML (zob. rysunek \ref{fig:ucased}). Podpis umieszcza się pod rysunkiem.

 
  % dokumentacja programistyczna
% !TeX spellcheck = pl_PL
% W szczególności rozdział ten powinien znaleźć się u osób 
% ze specjalności  BSI
\newpage\section{Bezpieczeństwo systemu \textsl{\NazwaSys}} \label{sec:bezpieczenstwo}
\subsection[Techniki kryptograficzne]{Techniki kryptograficzne [\StudentB]}
W systeme \textsl{\NazwaSys} zaimplementowano szereg funkcji kryptograficznych. Komunikacja pomiędzy modułami wymienionymi w punkcie XXX odbywa się przy pomocy Web API. Sama transmisja danych oparta jest o protokół HTTPs. Po stronie serwera przy pomocy paczek ''pySSl'' oraz ''Werkzug'' została zaimplementowana funkcja SSL dla wszystkich API włącznie z stroną internetową. Z racji braku posiadania własnego certyfikatu SSL w projekcie tym został wygenerowany przy pomocy paczki ''Werkzug'' przykładowy certyfikat.Z racji tego że ten urząd nie jest rozpoznawalny wyświetlają się komunikaty na stronie internetowej o niebezpieczeństwie w postaci niezaufanego certyfikatu. W aplikacji mobilnej na czas korzystania z niego została zaimplementowana opcja zezwalająca na używanie certyfikatów z niezaufanego źródła. Podczas wdrażania projektu w prawdziwych warunkach zalecamy wykorzystanie prawdziwego certyfikatu.

Funkcje kryptograficzne użyte w aplikacji mobilnej należy rozdzielić na dwie sekcje.Pierwszą z nich są funkcje używane podczas przechowywania danych użytkownika. Wszystkie newralgiczne informacje są szyfrowane w celu polepszenia poufności w systemie. Dane przechowywane w pamięci telefonu szyfrowane są funkcją kryptograficzną AES w trybie blokowym CBC. W przypadku aplikacji mobilnej korzystaliśmy z gotowej implementacji zawartej w bibliotece ''crypto'' przeznaczonej dla języka Java. Dane szyfrowane są trzema rodzajami haseł i tak hasło użytkownika przechowywane jest w pamięci telefonu (SharedPreferences) w postaci zaszyfrowanego tekstu.  ten jest szyfrowany hasłem zaszytym w implementacje aplikacji. Klucz szyfrujący oraz token szyfrowane są za pomocą hasła użytkownika. Podczas eksportu klucza szyfrującego wraz z certyfikatem klucza szyfrującego które łączone są w jeden plik   występuje dodatkowe szyfrowanie hasłem które użytkownik wpisze. Zapewnia to zwiększoną poufność oraz poprawia bezpieczeństwo podczas przenoszenia pliku.                

Drugą sekcją są funkcje kryptograficzne użyte podczas generowania klucza szyfrującego oraz używania go do podpisów. Algorytmem użytym do tego jest RSA. Sygnatura wykorzystywana w tym algorytmie to SHA-256. Do implementacji tych funkcji skorzystaliśmy z gotowej biblioteki ''security''. Klucze te są podstawą kryptograficzną naszego projektu oraz implementacji PKI. Generowane są one po stronie aplikacji mobilnej i z pary kluczy prywatny publiczny tylko publiczny jest przesyłany do serwera, i dzięki temu oraz polityce przechowywania tego klucza uniemożliwiamy przejęcie tego klucza przez osoby niepowołane.     

"z

\newpage
\subsection[Podatności systemu (OWASP Top 10)]{Podatności systemu (OWASP Top 10) \newline [\StudentA]}

\newpage
\subsection[Inne zagrożenia występujące w systemie]{Inne zagrożenia występujące w systemie \newline [\StudentB]}

\newpage
\subsection[Możliwości zabezpiezpieczenia systemu]{Możliwości zabezpiezpieczenia systemu \newline [\StudentA]}

\newpage
\subsection{Wnioski}
 
 % analiza aspektów bezpieczeństwa systemu
% !TeX spellcheck = pl_PL
\newpage\section{Wdrożenie i testowanie systemu \NazwaSys} \label{sec:testy}
\subsection{Środowisko testowe [\StudentA]}
% parametry komputera: procesor, liczba rdzeni, wielkość pamięci, karta graficzna, karta sieciowa, system operacyjny itp.

\subsection{Wizualizacja działania systemu \textsl{\NazwaSys}}
% przedstawic graficznie struktury danych, dane i wyniki działania aplikacji 
% można wykorzystać SVG

 
          % testy i dokumentacja użytkowa 
\newpage\section{Podsumowanie} \label{sec:podsumowanie}
	\subsection{Dalsze perspektywy rozwoju proejktu}
   % podsumowanie i wnioski
\newpage
% Literatura 
   \begin{thebibliography}{99}
    
	 \end{thebibliography}
 
     % literatura i źródłowe strony internetowe 
% !TeX spellcheck = pl_PL
%Spis rysunków i tabel
\newpage\listoffigures
  \addcontentsline{toc}{section}{Spis rysunków}
\listoftables
  \addcontentsline{toc}{section}{Spis tabel}
 
          % Spis rysunkow i tabel
% !TeX spellcheck = pl_PL
%\newpage\section{Dodatek} 
\newpage
\section*{Dodatki} \label{Dodatki}
\addcontentsline{toc}{section}{Dodatki}
\subsection*{Instalacja systemu \NazwaSys}
Instalacje systemu należy podzielić na 4 kroki. Pierwszym z nich jet instalacja serwera. Żeby wykonać to, należy zainstalować w podanej kolejności następujące elementy w systemie Linux bądź Windows:
\begin{itemize*}
	\item środowisko Python 2.7 wraz z programem pip,
	\item biblioteka pyCrypto zainstalowane przy pomocy polecenia pip -install pyCrypto,
	\item  framework Django przy pomocy polecenia pip -install Django,
	\item biblioteka MySQLdb przy pomocy polecenia pip -install MySQLdb,
	\item  biblioteka pySSL przy pomocy polecenia pip -install pySSL,
	\item biblioteka Werkzug przy pomocy polecenia pip - install Werkzug.
\end{itemize*}



W celu uruchomienia serwera należy podać polecenie \textit{python manage.py runserver\_plus xxx.xxx.xxx.xxx:8080 --cert-file cert.crt} gdzie w miejscu ''xxx.xxx.xxx.xxx'' należy wpisać adres IP serwera.

Instalacja urządzenia sterującego i moduły zliczania osób polega na pobraniu systemu raspian z strony internetowej \href{https://www.raspberrypi.org/downloads/}{https://www.raspberrypi.org/downloads/} oraz zainstalować odpowiednie biblioteki Pythona 2.7:
\begin{itemize*}
	\item biblioteka BlueZ, poprzez polecenie konsoli linuxowej: pip -install BlueZ,
	\item biblioteka PyCrypto, polecenie pip -install PyCrypto,
	\item biblioteka httplib, polecenie pip -install httplib,
	\item biblioteka OpenCV, polecenie pip -install python-opencv.
\end{itemize*}

Pobrane pliki z skryptem należy uruchomić poleceniami odpowiednio \textit{python Inteligentny\_zamek.py}, Wcześniej należy otworzyć plik w notatniku i podać adres IP aplikacji serwerowej opisanej wyżej. Moduł zliczania osób uruchamia się poleceniem \textit{python counter.py -ip 1} dodatek -ip 1 oznacza, że zostanie uruchomiona wersja programu z kamerą IP, bez tego parametru, domyślna kamera Raspberry Pi (jeśli taka jest podłączona).

Ostatnim elementem potrzebnym do funkcjonowania projektu jest aplikacja mobilna. W celu zainstalowania jej potrzebny będzie telefon mobilny z wersją systemu operacyjnego android w wersji minimalnej 5.0 z włączoną funkcją programistyczną  W celu wgrania aplikacji na telefon należy przekopiować plik z rozszerzeniem apk do pamięci telefonu a następnie uruchomić go w urządzeniu mobilnym (należy wcześniej zezwolić na instalowanie oprogramowania z nieznanego źródła).
 
\subsection*{Instrukcja użytkownika systemu \NazwaSys}

        % Dodatek lub dodatki
\newpage\section{Załączniki}
%\newpage\textbf{Załączniki}\\
%\addcontentsline{toc}{section}{Załączniki}
Do pracy dołączono płytę CD-ROM zawierającą:
\begin{itemize}
	\item treść pracy w pliku PDF,
	\item treść pracy w formacie LATEX,
	\item implementację systemu \NazwaSys,
	\item kody uruchomieniowne systemu \NazwaSys. 
\end{itemize} 
      % Załącznik lub załączniki
\end{document}