% !TeX spellcheck = pl_PL
\newpage
\section{Wdrożenie i testowanie systemu \NazwaSys} \label{sec:testy}
Rozdział skupiać się będzie na przedstawieniu działania systemu oraz w jakim środowisku zostało testowane. Wizualizacja funkcjonowania modułów, będzie jednocześnie, krótkimi testami poprawności zaprojektowanych poszczególnych części implementacji. 

\subsection{Środowisko testowe}
Stanowisko testowe podczas implementacji projektu składa się z:
\begin{itemize*}
	\item laptopa z systemem Windows 10 EDU służącego, jako serwer (aplikacja serwerowa, baza danych),
	\item mikrokomputera Raspberry Pi 3 Model B, jako urządzenie sterujące oraz moduł zliczania osób,
	\item kamera IP Dahua DH-IPC-HDW2220RP-ZS, jako kamera zliczająca,
	\item 3 smartphony różnych modeli:
	\begin{enumerate*}
		\item Samsung S5 Neo (Android 6.0),
		\item Samsung S5 (Android 6.0),
		\item ZTE Blade A452 (Android 5.1).
	\end{enumerate*}
\end{itemize*}

Testowy laptop posiada następujące parametry:
\begin{itemize*}
	\item pamięć ram 32 GB,
	\item procesor Intel Core i7-6700HQ,
	\item dysk SSD NVMe o średniej prędkości odczytu 1,6GB/s i odczycie 920MB/s,
	\item karta graficzna Nvidia Quadro M2000M,
	\item karta sieciowa Intel Dual Band Wireless-AC 8260.
\end{itemize*}

\subsection{Wizualizacja działania systemu \textsl{\NazwaSys}}
Poniżej zostaną opisane poszczególne funkcje systemu wraz z opisem interakcji pomiędzy użytkownikiem oraz systemem. Podczas wizualizacji zakładamy żę cały system został poprawnie zainstalowany. Środowisko testowe zostało dokładnie opisane w poprzednim punkcie.

\begin{enumerate*}
	\item Logowanie (strona internetowa) \newline
	Test polegał na wykonaniu logowania błędnymi danymi (Rys. \ref{rys:Strona3} oraz sprawdzenia czy pomimo braku zalogowania można było uzyskać dostęp strony historii użycia zamków. \newline
	Wnioskiem z testów jest stwierdzenie poprawności komunikowania o błędnym logowaniu. Próbując przejść na stronę główna nie będąc zalogowany, zostajemy przekierowani do strony logowania.
\begin{figure}[ht!]
		\centering
		\includegraphics[width=8.5cm]{Obrazy/Strona3}
		\caption{Strona logowania -- walidacja hasła}
		\label{rys:Strona3}
\end{figure}

	\item Zliczanie osób: \newline
	 Test polegał na sprawdzeni poprawności zliczania osób w korytarzu, gdy przechodzą przez niego osoby w obie strony.
	
	Wnioski: Podczas testu wykryto prawidłową liczbę osób, które umownie ''weszły'' oraz ''wyszły''. Efekt widoczny jest na zrzutach ekranu z urządzenia sterującego. Początkowo liczniki osób wchodzących i wychodzący w lewym góry rogu ekranu wynoszą zero (Rys. \ref{rys:Kamera1}), taka sama wartość występuje przy wartości na stronie internetowej (Rys. \ref{rys:Strona1}). Rysunek \ref{rys:Kamera2} i \ref{rys:Strona2} przedstawia stan po ''wejściu'' do pomieszczenia. Działanie zliczania osób wychodzących przedstawia Rys. \ref{rys:Kamera3}.

	\begin{figure}[ht!]
		\vspace{-0.35cm}
	\begin{minipage}{0.3\textwidth}
		\includegraphics[width=\textwidth]{Obrazy/Kamera1}
		\caption{Stan początkowy testu zliczania osób }
		\label{rys:Kamera1}
	\end{minipage}
	\hspace{0.01\textwidth}
	\begin{minipage}{0.69\textwidth}
		\vspace{-1cm}
		\includegraphics[width=\textwidth]{Obrazy/Strona1}
		\caption{Stan początkowy testu zliczania osób}
		\label{rys:Strona1}
	\end{minipage}
	\end{figure}

	\begin{figure}[ht!]
		\vspace{-1.5cm}
	\begin{minipage}{0.69\textwidth}
		\includegraphics[width=\textwidth]{Obrazy/Strona2}
		\caption{Stan po ''wejściu'' osoby }
		\label{rys:Strona2}
	\end{minipage}
	\hspace{0.01\textwidth}
	\begin{minipage}{0.3\textwidth}
		\includegraphics[width=0.9\textwidth]{Obrazy/Kamera2}
		\caption{Stan po ''wejściu'' osoby}
		\label{rys:Kamera2}
	\end{minipage}
\end{figure}

	\begin{figure}[ht!]
		\vspace{-1.7cm}
		\centering
		\includegraphics[width=4cm]{Obrazy/Kamera3}
		\caption{Stan po ''wyjściu'' osoby}
		\label{rys:Kamera3}
	\end{figure}
\newpage
	\item M Symulacja otwarcia
	\item M (strona) historia
	\item Logowanie użytkownika: \newline
	Test: Test polega na weryfikacji działania autoryzacji danych logowania oraz przy tym przydział odpowiednich uprawnień (administrator/zwykły użytkownik). Pierwsza próba logowania następuje z podaniem błędnego hasła użytkownika, następnie poprawnego (użytkownik TestowyZwykly) posiada zwykłe uprawnienia, trzecia próba związana będzie z zmianą uprawnień danego użytkownika na uprawnienia administratora.\newline
	Wnioski: Autoryzacja przedstawiona została na ilustracji Rys. \ref{rys:Logodwanie_blad_hasla}. Na ekranie pojawia się czerwony napis ''Błędny login lub hasło'', oznacza to w tym wypadku, że zostało podane błędne hasło (dany użytkownik istnieje w bazie danych, ale o innym haśle). Wyświetlane informacje nie wskazują na to, które dane zostały podane błędnie, przez co potencjalny włamywacz ma utrudnione zadanie przejęcia konta. Druga próba logowania skutkuje poprawnym zalogowaniem, jako użytkownik standardowy. Pojawia się okno główne, które po rozwinięciu menu wskazuje widok \ref{rys:panel_boczny_pionowo} (brak udostępnionych funkcji administracyjnych widocznych na rysunku \ref{rys:panel_boczny_pionowo2}). Trzecia próba logowania skutkuje pojawieniem się widoku \ref{rys:panel_boczny_pionowo2}, czyli zostały poprawnie przydzielone uprawnienia.
	
	\begin{figure}[ht!]
		\centering
		\includegraphics*[width=5cm]{Obrazy/APK_logowanie_blad}
		\caption{Logowanie użytkownika -- błąd hasła lub loginu}
		\label{rys:Logodwanie_blad_hasla}
	\end{figure}
	\item M rejestracja
	wyświetlanie podpowiedzi do hasła
	walidacja hasła 
	ukazywanie ukrywanie hasła
	rejestracja (komunikat +przekierowanie)
	\item M (admin)zaakceptowanie rejestracji
	widok rejestracji odrzucenie 
	akceptowanie wyniki
	\item D wnioskowanie o certyfikat
	widok wnioskowania o certyfikat
	\item D wygenerowanie nowego klucza dostępowego
	wygenerowanie nowego klucza (wnioski) 
	\item D zaakceptowanie certyfikatu dostępowego przez administratora
	akceptacja certyfikatu opusprzejscia
	\item D (admin) wygenerowanie certyfikatu dostępowego
	\item D  pobranie certyfikatu
	\item M uzyskanie dostępu do zamka (aktywacja bluetooth)
		 jako podpunkt pokazanie braku dostępu do zamka (zmiana daty)
	\item D przedłużenie certyfikatu dostępowego
	przypis bo opisane wcześniej
	\item M blokowanie certyfikatu szyfrującego *admin, user)
	\item M (admin) historia użycia zamka
	ukazanie hsitroi zamka
	\item D zarządzanie kontami użytkowników
			przypis bo wczesniej było opisane
			pod pkt zablokowanie klucza prywatnego
			przypis bo opisane wczesniej było
			pod pkt zablokowanie konta użytkownika
	\item  D eksport/import klucza
	przypis bo opisane w czesnije było
\end{enumerate*}
