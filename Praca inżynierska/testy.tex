% !TeX spellcheck = pl_PL
\newpage
\section{Wdrożenie i testowanie systemu \NazwaSys} \label{sec:testy}
Rozdział skupiać się będzie na przedstawieniu działania systemu oraz w jakim środowisku zostało testowane. Wizualizacja funkcjonowania modułów, będzie jednocześnie, krótkimi testami poprawności zaprojektowanych poszczególnych części implementacji. 

\subsection{Środowisko testowe}
Stanowisko testowe podczas implementacji projektu składa się z:
\begin{itemize*}
	\item laptopa z systemem Windows 10 EDU służącego, jako serwer (aplikacja serwerowa, baza danych),
	\item mikrokomputera Raspberry Pi 3 Model B, jako urządzenie sterujące oraz moduł zliczania osób,
	\item kamera IP Dahua DH-IPC-HDW2220RP-ZS, jako kamera zliczająca,
	\item 3 smartphony różnych modeli:
	\begin{enumerate*}
		\item Samsung S5 Neo (Android 6.0),
		\item Samsung S5 (Android 6.0),
		\item ZTE Blade A452 (Android 5.1).
	\end{enumerate*}
\end{itemize*}

Testowy laptop posiada następujące parametry:
\begin{itemize*}
	\item pamięć ram 32 GB,
	\item procesor Intel Core i7-6700HQ,
	\item dysk SSD NVMe o średniej prędkości odczytu 1,6GB/s i odczycie 920MB/s,
	\item karta graficzna Nvidia Quadro M2000M,
	\item karta sieciowa Intel Dual Band Wireless-AC 8260.
\end{itemize*}

\subsection{Wizualizacja działania systemu \textsl{\NazwaSys}}
Poniżej zostaną opisane poszczególne funkcje systemu wraz z opisem interakcji pomiędzy użytkownikiem oraz systemem. Podczas wizualizacji zakładamy żę cały system został poprawnie zainstalowany. Środowisko testowe zostało dokładnie opisane w poprzednim punkcie.

\begin{enumerate*}
	\item Logowanie (strona internetowa) \newline
	Test polegał na wykonaniu logowania błędnymi danymi (Rys. \ref{rys:Strona3} oraz sprawdzenia czy pomimo braku zalogowania można było uzyskać dostęp strony historii użycia zamków. \newline
	Wnioskiem z testów jest stwierdzenie poprawności komunikowania o błędnym logowaniu. Próbując przejść na stronę główna nie będąc zalogowany, zostajemy przekierowani do strony logowania.
\begin{figure}[ht!]
		\centering
		\includegraphics[width=8.5cm]{Obrazy/Strona3}
		\caption{Strona logowania -- walidacja hasła}
		\label{rys:Strona3}
\end{figure}

	\item Zliczanie osób: \newline
	 Test polegał na sprawdzeni poprawności zliczania osób w korytarzu, gdy przechodzą przez niego osoby w obie strony.
	
	Wnioski: Podczas testu wykryto prawidłową liczbę osób, które umownie ''weszły'' oraz ''wyszły''. Efekt widoczny jest na zrzutach ekranu z urządzenia sterującego. Początkowo liczniki osób wchodzących i wychodzący w lewym góry rogu ekranu wynoszą zero (Rys. \ref{rys:Kamera1}), taka sama wartość występuje przy wartości na stronie internetowej (Rys. \ref{rys:Strona1}). Rysunek \ref{rys:Kamera2} i \ref{rys:Strona2} przedstawia stan po ''wejściu'' do pomieszczenia. Działanie zliczania osób wychodzących przedstawia Rys. \ref{rys:Kamera3}.

	\begin{figure}[ht!]
		\vspace{-0.35cm}
	\begin{minipage}{0.3\textwidth}
		\includegraphics[width=\textwidth]{Obrazy/Kamera1}
		\caption{Stan początkowy testu zliczania osób }
		\label{rys:Kamera1}
	\end{minipage}
	\hspace{0.01\textwidth}
	\begin{minipage}{0.69\textwidth}
		\vspace{-1cm}
		\includegraphics[width=\textwidth]{Obrazy/Strona1}
		\caption{Stan początkowy testu zliczania osób}
		\label{rys:Strona1}
	\end{minipage}
	\end{figure}

	\begin{figure}[ht!]
		\vspace{-1.5cm}
	\begin{minipage}{0.69\textwidth}
		\includegraphics[width=\textwidth]{Obrazy/Strona2}
		\caption{Stan po ''wejściu'' osoby }
		\label{rys:Strona2}
	\end{minipage}
	\hspace{0.01\textwidth}
	\begin{minipage}{0.3\textwidth}
		\includegraphics[width=0.9\textwidth]{Obrazy/Kamera2}
		\caption{Stan po ''wejściu'' osoby}
		\label{rys:Kamera2}
	\end{minipage}
\end{figure}

	\begin{figure}[ht!]
		\vspace{-1.7cm}
		\centering
		\includegraphics[width=4cm]{Obrazy/Kamera3}
		\caption{Stan po ''wyjściu'' osoby}
		\label{rys:Kamera3}
	\end{figure}
\newpage
	\item M Symulacja otwarcia
	\item M (strona) historia
	\item M logowanie użytkownika
	Walidacja hasła 
	zalogowanie, 
	opisanie panelu bocznego
	\item M rejestracja
	wyświetlanie podpowiedzi do hasła
	walidacja hasła 
	ukazywanie ukrywanie hasła
	rejestracja (komunikat +przekierowanie)
	\item M (admin)zaakceptowanie rejestracji
	widok rejestracji odrzucenie 
	akceptowanie wyniki
	\item  wnioskowanie o certyfikat\newline
	Test polegał na sprawdzeniu listy z oczekującymi cretyfikatami, wykonaniu wnioskowania o certyfikat a następnie ponownym sprawdzeniu listy z oczekującymi certyfikatami.
	
	Wnioski: W stanie początkowym (Ry.s \ref{rys:wnioskowanie1}) nie było żadnego certyfikatu oczekujaćego na liście. Podczas wnioskowania o certyfikat w momencie włączzenia widoku (Ry.s \ref{rys:wnioskowanie2}) wyświetlił się  sie komunikat o treści ''pobrano listę zamków''. W kolejnym kroku podczas nacisnięcia na pole z zzamkiem (Ry.s \ref{rys:wnioskowanie3}) wyświetliła się wiadomość o treści "Wniosek zosatał wysłany". Po ponownym sprawdzeniu na liście oczekujących certyfikatów (Ry.s \ref{rys:wnioskowanie4}) znajdował się wniosek o certyfikat. Test zakońćzył się pomyślnie dla tego scenariusza.
	
	
	\begin{figure}[ht!]

		\begin{minipage}{0.2\textwidth}
			\includegraphics[width=\textwidth]{Obrazy/wnioskowanie1}
			\caption{Stan początkowy listy oczekujących certyfikatów na zaakceptowanie }
			\label{rys:wnioskowanie1}
		\end{minipage}
		\begin{minipage}{0.2\textwidth}
			\includegraphics[width=\textwidth]{Obrazy/wnioskowanie2}
			\caption{Stan początkowy podczas załadowania widoku wnioskowania o certyfikat}
			\label{rys:wnioskowanie2}
		\end{minipage}
	
		\begin{minipage}{0.2\textwidth}
		\includegraphics[width=\textwidth]{Obrazy/wnioskowanie3}
		\caption{Wnioskowanie o certyfikat}
		\label{rys:wnioskowanie3}
	\end{minipage}
	\begin{minipage}{0.2\textwidth}
		\includegraphics[width=\textwidth]{Obrazy/wnioskowanie4}
		\caption{Stan listy opczekujących certyfikatów po wnioskowaniu o certyfikat}
		\label{rys:wnioskowanie4}
	\end{minipage}
	\end{figure}
	
	
	
	
	\item wygenerowanie nowego klucza dostępowego\newline
	Test polegał na sprawdzeniu wartośći certyfikatu klucza dostępowego, wygenrowaniu nowego i spradzenie ponowne wartośći certyfikatu dostępowego.
	Wnioski: Podczas testu w stanie początkowym (Ry.s \ref{rys:generowanieKD1})znajdował się certyfikat dostępowy ważny do 2019.01.26 14:23:55. Po naciśnieciu przycisku ''Wygeneruj''  (Ry.s \ref{rys:generowanieKD2}) certyfikat został zmeinion na nowy z datą ważności o godzinę dłuższą.
 
	
	\begin{figure}[ht!]
		
		\begin{minipage}{0.4\textwidth}
			\includegraphics[width=\textwidth]{Obrazy/generowanieKD1}
			\caption{Stan początkowy wyświetlonego certygikatu klucza dostępowego }
			\label{rys:generowanieKD1}
		\end{minipage}
		\begin{minipage}{0.4\textwidth}
			\includegraphics[width=\textwidth]{Obrazy/generowanieKD2}
			\caption{Stan po wygenerowaniu nowego certyfikatu klucza dostępowego}
			\label{rys:generowanieKD2}
		\end{minipage}
		
	
	\end{figure}
	
	
	\item  zaakceptowanie certyfikatu dostępowego przez administratora
	Test polegał na przejśćiu do widoku odpowiedzialnego za zarządzanie oczekującymi certyfikatami dostępu oraz odpowiednio usunięcie i zaakceptowanie wbniosku:
	Wnioski: w stanie początkowym znajdowały się dwa certyfikaty (Ry.s \ref{rys:generowanieCert1}. Po nacisnięciu przycisku usuń z listy zniknął dany wniosek (Ry.s \ref{rys:generowanieCert2}. Po wyborze akceptuj zostaliśmy przekierowani do widoku generowania certyfikatu. Po ponownym wejśćiu do listy certyfikatów (Ry.s \ref{rys:generowanieCert3}) lista ta była pusta. Test został pomyślnie przeprowadzony (Ry.s \ref{rys:generowanieCert4}). 
	
		\begin{figure}[ht!]
		
		\begin{minipage}{0.2\textwidth}
			\includegraphics[width=\textwidth]{Obrazy/oczekujaceCert0}
			\caption{Stan początkowy listy oczekujących certyfikatów na zaakceptowanie }
			\label{rys:generowanieCert1}
		\end{minipage}
		\begin{minipage}{0.2\textwidth}
			\includegraphics[width=\textwidth]{Obrazy/oczekujaceCert1}
			\caption{Stan listy oczekujaćych certyfikatów po usunięciu z listy elementu}
			\label{rys:generowanieCert2}
		\end{minipage}
		
		\begin{minipage}{0.2\textwidth}
			\includegraphics[width=\textwidth]{Obrazy/oczekujaceCert2}
			\caption{Stan listy oczekujaćych certyfikatów po ponownym otworzeniu}
			\label{rys:generowanieCert3}
		\end{minipage}
	
	\begin{minipage}{0.2\textwidth}
		\includegraphics[width=\textwidth]{Obrazy/generowanieerror}
		\caption{Stan listy oczekujaćych certyfikatów po ponownym otworzeniu}
		\label{rys:generowanieCert4}
	\end{minipage}
	
	\end{figure}
	
	
	
	\item  Wygenerowanie certyfikatu dostępowego
	Test polegą na przeprowadzeniu procesu generowania certyfikatu z uwzględniemiem wpisywania niepoprawnych danych.
	
	Wnioski W początkowym stanie zostały wypełnione dane użytkownika oraz wybrany został login i zamek (Ry.s \ref{rys:generowanie1}), Następnie po naciśnięciu przycisk dodaj zakresy zostało wykonane przejście do widoku  wygenerowania zakresów (Rys. \ref{rys:generowanie2}). W nim wprowadzono  błędny zakres (Rys. \ref{rys:generowanie3}). Została zwrócona walidacja (Rys. \ref{rys:generowanie4}). Następnie zostął dodany poprawne zakresy (Rys. \ref{rys:generowanie5}). Potem został usunięty jeden zakres (Rys. \ref{rys:generowanie6})   . Po naciśnięciu przycisku kontynuj generowanie certyfikatu zostało wykonane przekierowanie  do daljszej częśći generowania certyfikatu. W widoku tym pozostały wszystkie dane uzupełnione przed przejśćiem do dodawnia zakresów. Wprowadzony został błędny zakres dat . Została wyświetlona walidacja (Rys. \ref{rys:generowanie7}).Po poprawie danych proces został pozytywnie ukończony w postaci przejśćia do widoku panelu administratora.
	
	
	
		\begin{figure}[ht!]
		
		\begin{minipage}{0.2\textwidth}
			\includegraphics[width=\textwidth]{Obrazy/generowanie1}
			\caption{Stan początkowy listy oczekujących certyfikatów na zaakceptowanie }
			\label{rys:generowanie1}
		\end{minipage}
	
		
		\begin{minipage}{0.2\textwidth}
			\includegraphics[width=\textwidth]{Obrazy/generowanie3}
			\caption{Wnioskowanie o certyfikat}
			\label{rys:generowanie2}
		\end{minipage}
		\begin{minipage}{0.2\textwidth}
			\includegraphics[width=\textwidth]{Obrazy/generowanie4}
			\caption{Stan listy opczekujących certyfikatów po wnioskowaniu o certyfikat}
			\label{rys:generowanie3}
		\end{minipage}
	\end{figure}


	


	\begin{figure}[ht!]
	
		\begin{minipage}{0.2\textwidth}
		\includegraphics[width=\textwidth]{Obrazy/generowanieerror}
		\caption{Stan początkowy listy oczekujących certyfikatów na zaakceptowanie }
		\label{rys:generowanie4}
	\end{minipage}
	
		\begin{minipage}{0.2\textwidth}
		\includegraphics[width=\textwidth]{Obrazy/generowanie5}
		\caption{Stan początkowy listy oczekujących certyfikatów na zaakceptowanie }
		\label{rys:generowanie5}
	\end{minipage}
	

	\begin{minipage}{0.2\textwidth}
		\includegraphics[width=\textwidth]{Obrazy/generowanie6}
		\caption{Stan początkowy podczas załadowania widoku wnioskowania o certyfikat}
		\label{rys:generowanie6}
	\end{minipage}
	
	\begin{minipage}{0.2\textwidth}
		\includegraphics[width=\textwidth]{Obrazy/generowanie8}
		\caption{Wnioskowanie o certyfikat}
		\label{rys:generowanie7}
	\end{minipage}

\end{figure}
	
	
	\item   pobranie certyfikatu
	Test polegał na wyczyszczeniu pamięći telefonu. Sprawdzeniu listy certryfikatów pobraniu samcyh certyfikatów oraz ponownym sprawdzeniu listy cerrtyfikatów.
	Wnioski: W stanie początkowym na liśćie nie były wyświetlane żadne cerrtyfikaty (Rys. \ref{rys:getCert1}). Nastepnie zostało wybrane pole pobierz z serwera. W tym momencie został wyświetlony komun ikat ''pobrano z serwera''(Rys. \ref{rys:getCert2}). Po ponownym przejsciu do listy certyfikatów widniała już pozycja na liście (Rys. \ref{rys:getCert3}).
	
		\begin{figure}[ht!]
		
		\begin{minipage}{0.2\textwidth}
			\includegraphics[width=\textwidth]{Obrazy/getCer1}
			\caption{Stan początkowy listy oczekujących certyfikatów na zaakceptowanie }
			\label{rys:getCert1}
		\end{minipage}
		
		\begin{minipage}{0.2\textwidth}
			\includegraphics[width=\textwidth]{Obrazy/getCert2}
			\caption{Stan początkowy listy oczekujących certyfikatów na zaakceptowanie }
			\label{rys:getCert2}
		\end{minipage}
		
		
		\begin{minipage}{0.2\textwidth}
			\includegraphics[width=\textwidth]{Obrazy/getCert3}
			\caption{Stan początkowy podczas załadowania widoku wnioskowania o certyfikat}
			\label{rys:getCert3}
		\end{minipage}
		
	
	\end{figure}
	
	
	\item M uzyskanie dostępu do zamka (aktywacja bluetooth)
		 jako podpunkt pokazanie braku dostępu do zamka (zmiana daty)
	\item D przedłużenie certyfikatu dostępowego
	Test polegał na nacisnieciu przycisku usun oraz przedłuż w widoku certyfikatu (Rys. \ref{rys:przedluzCert1}).
	Wnioski dla konta administratora usuń certyfikat powodowało usunięcie go z listy i przejśćie do widoku certyfikatów. Przycisk przedłuż powodował przeniesienie do widoku generowania. Dla konta o uprawnieniach  mniejszych od administratora przycisk usuń działał analogicznie jak u administratora natomiast przycisk przedłuż powodował wyświeltenie komunikatu ''wysłano wniosek'' (Rys. \ref{rys:przedluzCert2}).
	Po sprawdzeniu w widoku lista oczekujacych certyfikatow po wnioskowniau widniaj on na liscie. Test zotał pomyślnie zakońćzony
	
	
		\begin{figure}[ht!]
		
		\begin{minipage}{0.2\textwidth}
			\includegraphics[width=\textwidth]{Obrazy/przedluzCert1}
			\caption{Stan początkowy listy oczekujących certyfikatów na zaakceptowanie }
			\label{rys:przedluzCert1}
		\end{minipage}
		
		\begin{minipage}{0.2\textwidth}
			\includegraphics[width=\textwidth]{Obrazy/przedluzCert2}
			\caption{Stan początkowy listy oczekujących certyfikatów na zaakceptowanie }
			\label{rys:przedluzCert2}
		\end{minipage}
		
		
	
		
	\end{figure}
	
	
	\item M blokowanie certyfikatu szyfrującego *admin, user)
	\item M (admin) historia użycia zamka
	ukazanie hsitroi zamka
	\item D zarządzanie kontami użytkowników
		Test polegał na wyświetleniu listy użytkownikó zablokowaniu klucza dostępowego oraz konta.
		
		Wnioski: Podczas testu w stanie początkowym wszyscy uzytkownicy systemu posiadali aktualny klucz dostępowy oraz nie byli zablokowani (Rys. \ref{rys:zarzadzanieKontem1}). Po zablokowniu klucza dostępowego  Administratora zmienił sie status na liśćie(Rys. \ref{rys:zarzadzanieKontem3}). Po Zablokowaniu konta użytkownika(Rys. \ref{rys:zarzadzanieKontem4}) również zmienił się status na liście (Rys. \ref{rys:zarzadzanieKontem5}). po wygenerowaniu ponownie klucza dostępowego nastąpiła zmiana statusu na liśćie (Rys. \ref{rys:zarzadzanieKontem6}). Próba zalogowania się na koncie zablokowanym nie doszla do skutku. Został wyświetlony komunikat ''Konto zostało zablokowane'' (Rys. \ref{rys:zarzadzanieKontem7}). Test przebiegł pomyślnie
		
		
				\begin{figure}[ht!]
					\begin{minipage}{0.2\textwidth}
						\includegraphics[width=\textwidth]{Obrazy/konto1}
						\caption{Stan początkowy listy oczekujących certyfikatów na zaakceptowanie }
						\label{rys:zarzadzanieKontem1}
					\end{minipage}
				\begin{minipage}{0.2\textwidth}
					\includegraphics[width=\textwidth]{Obrazy/konto2}
					\caption{Stan początkowy listy oczekujących certyfikatów na zaakceptowanie }
					\label{rys:zarzadzanieKontem2}
				\end{minipage}
			
			\begin{minipage}{0.2\textwidth}
				\includegraphics[width=\textwidth]{Obrazy/konto3}
				\caption{Stan początkowy listy oczekujących certyfikatów na zaakceptowanie }
				\label{rys:zarzadzanieKontem3}
			\end{minipage}
			
			\begin{minipage}{0.2\textwidth}
				\includegraphics[width=\textwidth]{Obrazy/konto4}
				\caption{Stan początkowy listy oczekujących certyfikatów na zaakceptowanie }
				\label{rys:zarzadzanieKontem4}
			\end{minipage}
			
			
			\begin{minipage}{0.2\textwidth}
				\includegraphics[width=\textwidth]{Obrazy/konto5}
				\caption{Stan początkowy podczas załadowania widoku wnioskowania o certyfikat}
				\label{rys:zarzadzanieKontem5}
			\end{minipage}
		
		
			\begin{minipage}{0.2\textwidth}
			\includegraphics[width=\textwidth]{Obrazy/konto6}
			\caption{Stan początkowy podczas załadowania widoku wnioskowania o certyfikat}
			\label{rys:zarzadzanieKontem6}
		\end{minipage}
	
		\begin{minipage}{0.2\textwidth}
		\includegraphics[width=\textwidth]{Obrazy/konto7}
		\caption{Stan początkowy podczas załadowania widoku wnioskowania o certyfikat}
		\label{rys:zarzadzanieKontem7}
	\end{minipage}

			
		\end{figure}
		
	\item  D eksport/import klucza
	Test polegał na eksporcie oraz imporcie klucza szyfrująćego wraz z próbami niepoprawnego wpisywania hasła.
	
	Wnioski: Podczas póby ekspory z błędnie wprowadzonym hasłem została wyswietlona walidacja(Rys. \ref{rys:impExp1}) w postaci komunikatu ''niepoprawne hasło''. Po wprowadzeniu poprawnego hasłą oraz eksporcie został wyśewietlony komunikat (Rys. \ref{rys:impExp2} ''poprawnie wyeksportowano certyfikat''. Podczas próbuy importu z błędnie wprowadzonym hasłem został wyświertlony komunikat (Rys. \ref{rys:impExp1} ''niepoprawne hasło''. Po wprowadzeniu poprawnego hasła i ponownej próBie został wyświetlony komunikat (Rys. \ref{rys:impExp3} ''poprawnie zimportowano certyfikat''. Test przebiegł pomyślnie.
	
	
	
	\begin{figure}[ht!]
		\begin{minipage}{0.2\textwidth}
			\includegraphics[width=\textwidth]{Obrazy/impExp1}
			\caption{Stan początkowy listy oczekujących certyfikatów na zaakceptowanie }
			\label{rys:impExp1}
		\end{minipage}
		\begin{minipage}{0.2\textwidth}
			\includegraphics[width=\textwidth]{Obrazy/impExp2}
			\caption{Stan początkowy listy oczekujących certyfikatów na zaakceptowanie }
			\label{rys:impExp2}
		\end{minipage}
		
		\begin{minipage}{0.2\textwidth}
			\includegraphics[width=\textwidth]{Obrazy/impExp3}
			\caption{Stan początkowy listy oczekujących certyfikatów na zaakceptowanie }
			\label{rys:impExp3}
		\end{minipage}
		
		
		
		
	\end{figure}
	
\end{enumerate*}
