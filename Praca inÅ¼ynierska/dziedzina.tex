% 
\newpage\section{Opis dziedziny przedmiotowej pracy}\label{sec:dziedzina}
\subsection{Pojęcia i definicje}
W dokumencie tym posługiwać się będziemy następującymi pojęciami:


	\begin{itemize}
	\item \textbf{Klucz dostępowy} --- jest to klucz publiczny z pary kluczy prywartny publiczny. Używany jest on do odszyfrowania wiadomości wysłanej z aplikacji mobilnej do urządzenia sterujacego,
	
	\item \textbf{Klucz szyfrujący} 
	--- jest to klucz prywatny wygenerowany podczas tworzenia pary kluczy publiczny prywatny. Używane jest on do szyfrowania wiadomości wysyłanej z aplikacji mobilnej do urządzenia sterującego,
	
    \item \textbf{Para kluczy szyfrujących}
    --- jest to para kluczy (prywatny oraz publiczny) generowanych podczas rejestracji oraz wymiany klucza dostepowego,
	
	\item \textbf{Inteligentny zamek}
	--- system obsługujący otwieranie elektrozamka bądz serwomechanizmu,
    \item \textbf{Konto}
	--- reprezentacja użytkownika w systemie za pomocą takich danych jak login hasło, imie, nazwisko.
	\item \textbf{Administrator}
	---jest to fizyczna osoba posiadająća dla swojego konta uprawnienia administratora co wiąże się z pełnym dostępem do systemu.					
\end{itemize}



\newpage
\subsection{Stan wiedzy}
Przed przystąpieniem do projektu zrobiliśmy porównanie systemów zbliżonych do naszego który na dany moment istniały. I tak doszliśmy do wniosku że wszystkie systemy inteligentnych zamków wykonane przez firmy takie jak Gerda Lock czy Danalock zostały wykonane typowo dla użytku domowego a nie tak jak nasz projekt inżynierski który jest przeznaczony do zarządzania w budynkach o wielu pomieszczeniach z różnym stopniem dostępu. Opis wraz z porównaniem poszcególnych systemów znajduje się w tabelach poniżej.


	Tabela \ref{tab:porownanie1} zawiera porównanie firm pod względem otwierania zamka
	\begin{longtable}[!ht]{|p{4cm}|p{1,5cm}|p{1,5cm}|p{1,5cm}|p{1,5cm}|} 
		\caption{Tabela porównania otwierania zamków}
		\label{tab:porownanie1}\\
		\hline	
		 & NOKI & August & DanaLock & Gerda Lock  \\	\hline
		 zarządzanie wieloma zamkami z jednej aplikacji
		 & brak & tak & brak & brak \\	\hline
		otwieranie zamka przy pomocy strony WWW
		& brak & brak & tak & brak \\	\hline
		inne sposoby otwarcia zamka niż aplikacja
		& brak informacji& brak informacji & brak & tak \\	\hline
		automatyczne zamykanie zamka
		& brak informacji& tak & tak & tak   \\	\hline
		tryb otwierania zamka automatycznie
		& tak& brak & tak & tak \\	\hline	
		tryb otwierania zamka po zezwoleniu przyciskiem
		& brak & tak & tak & tak \\		\hline
	\end{longtable}


 
 	Tabela \ref{tab:porownanie2} zawiera porównanie firm pod względem zasilania i montażu
 \begin{longtable}[!ht]{|p{4cm}|p{1,5cm}|p{1,5cm}|p{1,5cm}|p{1,5cm}|} 
 	\caption{Tabela porównania zasialania i montażu}
 	\label{tab:porownanie2}\\
 	\hline	
 	& NOKI & August & DanaLock & Gerda Lock  \\	\hline
 	zasilanie zewnętrzne (z sieci)	
 	& brak & brak & brak & brak \\	\hline
	 zasilanie bateryjne (podstawowe/ awaryjne)	
	 & podstawowe & podstawowe & podstawowe & podstawowoe \\	\hline
 	sposób montażu	
 	& nakłądka na zamek & nakłądka na zamek & nakłądka na zamek & nakłądka na zamek \\	\hline
 \end{longtable}
 



Tabela \ref{tab:porownanie3} zawiera porównanie firm pod względem dziennika zdarzeń oraz powiadomień
\begin{longtable}[!ht]{|p{4cm}|p{1,5cm}|p{1,5cm}|p{1,5cm}|p{1,5cm}|} 
	\caption{Tabela porównania zasialania i montażu}
	\label{tab:porownanie3}\\
	\hline	
	& NOKI & August & DanaLock & Gerda Lock  \\	\hline
	podgląd kto otworzył	
	& brak informacji & brak & brak & tak \\	\hline
	
	
	powiadomienie o otwarciu drzwi (ogólnie i przez daną osobę)
	& brak & brak & brak & tak \\	\hline
	
	
	powiadomienie o nieautoryzowanych próbach otwarcia
	& tak & brak & brak & tak \\	\hline
\end{longtable}

\newpage
\subsection{Stan pracy wykonany w ramach zajęć \newline przedmiotowych} 
W ramach zajęć projektowych oraz laboratoryjnych o nazwe Projekt Zespołowy prowadzonych z mgr. Michałem Apolinarskim oraz dr Ewą Idzikowską zostały wykonane następujace fragmenty systemu:
	
	Aplikacja mobilna została wykonana dla wersji andorida minimum 4.4 KitKat w stopniu umożliwiającym takie funkcjonalnośći jak:
	\begin{itemize}
		\item Logowanie
		\item Rejestracja
		\item Rejestracja wraz z tworzeniem pary kluczy dostępowych publiczny prywatny
		\item Generowanie nowego certyfikatu
		\item Pobieranie certyfikatów z serwera
		\item Zarządznanie certyfikatami użytkownika
		\item Zarządzanie prośbami o rejestracje
		\item Wnioskowanie o certyfikat nowy
	\end{itemize}
		Dodatkowo zostało napisane api do obsługi połączenia bluetooth oraz w każdym widoku któy korzystał z połaczenia z serwerem były napisane fragmenty kodu. Funkcje te oraz kod zostały napisane bez uwzględnienia wzorców architektoniczncych (wszystko co dotyczyło danego widoku było w jednej klasie), posiadały szereg błędów powodujaćych niestabilne działąnie systemu oraz posiadały metody z systemu android które były określane przez środowisko android stuido jako "deprecated" co mogło przy nowszych wersjach androida powodować wadliwe działanie systemu. Z racji pisania pod wersje systemu android 4.4 wygląd różni się od tego który został zaimplementowany w pracy inżynierskiej.
		 
	
	
   Aplikacja serwerowa została wykonanan w stopniu umożliwiającym podstawowoe funkcje takie jak:
   		\begin{itemize}
   		\item Logowanie
   		\item Rejestracja
   		\item Zapisywanie nowego certyfikatu
   		\item udostępnianie certyfikatóW
   		\item pobieranie list certyfikatów, próśb o certyfikaty oraz rejestracji
   	\end{itemize}
	Wszystkie te funkcje zwracały odpowiednio albo pożądane dane albo wartość 'Invalid' co powodowało wyświetlania błędnych komunikaótw uzytkownikowi z racji nierozróżniania braku dostępu od błędnie wykonanego skyptu.
	
	Urządzenie sterujące zamkiem zostało napisane w języku python i pozwalało na odbieranie danych z aplikacji mobilnej oraz posiadało funckje odpowiedzialną za otwieranie zamka.
 

	W ramach przedmiotu ochrona danych zostały zaimplementowane w systemie fragmenty PKI takie jak:
	   \begin{itemize}
	   	\item Certyfikat klucza dostępowego
	   	\item  generowanei nowego Certyfikatu uzytkownika
	   	\item blokowanie uzytkownika systemu
	   \end{itemize}
	 Funkcje te zostały napisane zarówno po stronie aplikacji mobilnej jak i aplikacji serwerowej. Ponadto po stronie androida został opracowany sposób przechowywania klucza prywatnego w formie zaszyfrowanego pliku hasłem uzytkownika.
