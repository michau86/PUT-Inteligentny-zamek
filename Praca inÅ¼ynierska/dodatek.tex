% !TeX spellcheck = pl_PL
%\newpage\section{Dodatek} 
\newpage
\section*{Dodatki} \label{Dodatki}
\addcontentsline{toc}{section}{Dodatki}
\subsection*{Instalacja systemu \NazwaSys}
Instalacje systemu należy podzielić na 4 kroki.
W pierwszym korku należy zainstalować serwer bazy danych MySQL. Można to wykonać za pomocą instalacji pakietu xampp w którym zawiera się ten serwer baz danych. W kolejnym etapie należy utworzyć bazę danych o nazwie ''Inteligentny\_zamek\_db'' a następnie zaimportować do niej plik z bazą danych o nazwie ''baza\_danych.db''. W tym momencie mamy utworzoną bazę danych wraz z kontem administratora o loginie ''administrator'' i haśle ''Admin''.  

 W tym momencie przechodzimy do instalacja serwera. Żeby wykonać to, należy zainstalować w podanej kolejności następujące elementy w systemie Linux bądź Windows:
\begin{itemize*}
	\item środowisko Python 2.7 wraz z programem pip,
	\item biblioteka pyCrypto zainstalowane przy pomocy polecenia pip -install pyCrypto,
	\item  framework Django przy pomocy polecenia pip -install Django,
	\item biblioteka MySQLdb przy pomocy polecenia pip -install MySQLdb,
	\item  biblioteka pySSL przy pomocy polecenia pip -install pySSL,
	\item biblioteka Werkzug przy pomocy polecenia pip - install Werkzug.
\end{itemize*}



W celu uruchomienia serwera należy podać polecenie:\newline
\textit{python manage.py runserver\_plus xxx.xxx.xxx.xxx:8080 --cert-file cert.crt}, \newline
gdzie w miejscu ''xxx.xxx.xxx.xxx'' należy wpisać adres IP serwera.

W tym momencie należy utworzyć konto superuser-a dla administratora który będzie mieć dostęp do strony internetowej z historią. utworzenie tego kont odbywa się przy pomocy polecenia ''python manage.py createsuperuser''. Po wpisaniu tego polecenia należy podać login adres e-mail oraz hasło wraz z powtórzeniem tego hasła. po poprawnej konfiguracji mamy już dostęp do strony internetowej.

Instalacja urządzenia sterującego i moduły zliczania osób polega na pobraniu systemu raspian z strony internetowej \href{https://www.raspberrypi.org/downloads/}{https://www.raspberrypi.org/downloads/} oraz zainstalować odpowiednie biblioteki Pythona 2.7:
\begin{itemize*}
	\item biblioteka BlueZ, poprzez polecenie konsoli linuxowej: pip -install BlueZ,
	\item biblioteka PyCrypto, polecenie pip -install PyCrypto,
	\item biblioteka httplib, polecenie pip -install httplib,
	\item biblioteka OpenCV, polecenie pip -install python-opencv.
\end{itemize*}

Pobrane pliki z skryptem należy uruchomić poleceniami odpowiednio \textit{python Inteligentny\_zamek.py}, Wcześniej należy otworzyć plik w notatniku i podać adres IP aplikacji serwerowej opisanej wyżej. Moduł zliczania osób uruchamia się poleceniem \textit{python counter.py -ip 1} dodatek -ip 1 oznacza, że zostanie uruchomiona wersja programu z kamerą IP, bez tego parametru, domyślna kamera Raspberry Pi (jeśli taka jest podłączona).

Ostatnim elementem potrzebnym do funkcjonowania projektu jest aplikacja mobilna. W celu zainstalowania jej potrzebny będzie telefon mobilny z wersją systemu operacyjnego android w wersji minimalnej 5.0 z włączoną funkcją programistyczną  W celu wgrania aplikacji na telefon należy przekopiować plik z rozszerzeniem apk do pamięci telefonu a następnie uruchomić go w urządzeniu mobilnym (należy wcześniej zezwolić na instalowanie oprogramowania z nieznanego źródła).

