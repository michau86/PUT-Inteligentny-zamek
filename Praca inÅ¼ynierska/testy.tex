% !TeX spellcheck = pl_PL
\newpage
\section{Wdrożenie i testowanie systemu \NazwaSys} \label{sec:testy}
Rozdział skupiać się będzie na przedstawieniu działania systemu oraz w jakim środowisku zostało testowane. Wizualizacja funkcjonowania modułów, będzie jednocześnie, krótkimi testami poprawności zaprojektowanych poszczególnych części implementacji. 

\subsection{Środowisko testowe}
Stanowisko testowe podczas implementacji projektu składa się z:
\begin{itemize*}
	\item laptopa z systemem Windows 10 EDU służącego, jako serwer (aplikacja serwerowa, baza danych),
	\item mikrokomputera Raspberry Pi 3 Model B, jako urządzenie sterujące oraz moduł zliczania osób,
	\item kamera IP Dahua DH-IPC-HDW2220RP-ZS, jako kamera zliczająca,
	\item 3 smartphony różnych modeli:
	\begin{enumerate*}
		\item Samsung S5 Neo (Android 6.0),
		\item Samsung S5 (Android 6.0),
		\item ZTE Blade A452 (Android 5.1).
	\end{enumerate*}
\end{itemize*}

Testowy laptop posiada następujące parametry:
\begin{itemize*}
	\item pamięć ram 32 GB,
	\item procesor Intel Core i7-6700HQ,
	\item dysk SSD NVMe o średniej prędkości odczytu 1,6GB/s i odczycie 920MB/s,
	\item karta graficzna Nvidia Quadro M2000M,
	\item karta sieciowa Intel Dual Band Wireless-AC 8260.
\end{itemize*}

\subsection{Wizualizacja działania systemu \textsl{\NazwaSys}}
Poniżej zostaną opisane poszczególne funkcje systemu wraz z opisem interakcji pomiędzy użytkownikiem oraz systemem. Podczas wizualizacji zakładamy żę cały system został poprawnie zainstalowany. Środowisko testowe zostało dokładnie opisane w poprzednim punkcie.

\begin{enumerate*}
	\item zliczanie osób w pomieszczeniu
	\item (strona) wyświetlenie
	\item (strona) historia
	\item rejestracja
	wyświetlanie podpowiedzi do hasła
	walidacja hasła 
	ukazywanie ukrywanie hasła
	rejestracja (komunikat +przekierowanie)
	\item logowanie admin
	Walidacja hasła 
	zalogowanie, 
	opisanie panelu bocznego
	\item (admin) zmiana hasła
	walidacja nowego hasła
	zmiana hasła (niepoprawnie komunikat)
	zmiana hasła poprawnie
	\item (admin)zaakceptowanie rejestracji
	widok rejestracji odrzucenie 
	akceptowanie wyniki
	\item (admin)wylogowanie
	poikazanie poprawnosci wylogowania (opis gdzie przechoidiz
	\item wnioskowanie o certyfikat
	widok wnioskowania o certyfikat
	 
	\item wygenerowanie nowego klucza
	wygenerowanie nowego klucza (wnioski) 
	\item (admin) zaakceptowanie certyfikatu
	akceptacja certyfikatu opusprzejscia
	\item (admin) wygenerowanie certyfikatu
	\item  pobranie certyfikatu
	\item lista certyfikatów 
	\item uzyskanie dostępu do zamka
	\item  pokazanie braku dostępu do zamka (zmiana daty)
	\item przedłużenie certyfikatu
	przypis bo opisane wczesniej
	\item usuniecie certyfikatu
	przypis bo opisane wczesniej
	\item (admin) historia użycia zamka
	ukazanie hsitroi zamka
	\item  zarządzanie kontami użytkowników
	przypis bo wczesniej było opisane
	\item zablokowanie klucza prywatnego
	przypis bo opisane wczesniej było
	\item zablokowanie konta użytkownika
	\item  eksport klucza
	przypis bo opisane w czesnije było
	\item import klucza
	przypis bo opisane w zesniej było
	\item zmiana adresu IP
	przypi bo opisane wczesniej bYło
\end{enumerate*}
