% !TeX spellcheck = pl_PL
\newpage
\section{Wdrożenie i testowanie systemu \NazwaSys} \label{sec:testy}
Rozdział skupiać się będzie na przedstawieniu działania systemu oraz w jakim środowisku zostało testowane. Wizualizacja funkcjonowania modułów, będzie jednocześnie, krótkimi testami poprawności zaprojektowanych poszczególnych części implementacji. 

\subsection{Środowisko testowe}
Stanowisko testowe podczas implementacji projektu składa się z:
\begin{itemize*}
	\item laptopa z systemem Windows 10 EDU służącego, jako serwer (aplikacja serwerowa, baza danych),
	\item mikrokomputera Raspberry Pi 3 Model B, jako urządzenie sterujące oraz moduł zliczania osób,
	\item kamera IP Dahua DH-IPC-HDW2220RP-ZS, jako kamera zliczająca,
	\item 3 smartphony różnych modeli:
	\begin{enumerate*}
		\item Samsung S5 Neo (Android 6.0),
		\item Samsung S5 (Android 6.0),
		\item ZTE Blade A452 (Android 5.1).
	\end{enumerate*}
\end{itemize*}

Testowy laptop posiada następujące parametry:
\begin{itemize*}
	\item pamięć ram 32 GB,
	\item procesor Intel Core i7-6700HQ,
	\item dysk SSD NVMe o średniej prędkości odczytu 1,6GB/s i odczycie 920MB/s,
	\item karta graficzna Nvidia Quadro M2000M,
	\item karta sieciowa Intel Dual Band Wireless-AC 8260.
\end{itemize*}

\subsection{Wizualizacja działania systemu \textsl{\NazwaSys}}
