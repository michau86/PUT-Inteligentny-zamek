% !TeX spellcheck = pl_PL
\newpage
\thispagestyle{empty}
\begin{center}
Poznan University of Technology\\
Faculty of Electrical Engineering\\
Institute of Control, Robotics and Information Engineering\\
  \vspace{20mm}
\huge{\TytulAngielski} \\
\Large{by}\\
  \vspace{5mm}
\Large{\StudentA}\\
\Large{\StudentB}\\
  \vspace{20mm}

\normalsize\textbf{Abstract} \\
{Thesis presents the design and implementation of a traffic control system and room access management. \linebreak The basic assumption is to supervise the opening and closing of doors in the area of a large building where there are many entrances. The system stores data about users and their access rights, when and where they can enter. This information can be found in the MySQL database located on the server. The presented effects of the system are divided into a mobile application with the Android system and a website. The remaining part of the system (control devices, server) perform logic functions in the system, they connect strictly program modules between them. The virtual part of the system is based on Python 2.7 and Django.} 

\end{center}

\begin{center}
 \textbf{Streszczenie} \\
 {Praca dyplomowa przedstawia projekt i wykonanie systemu kontroli ruchu i zarządzania dostępem do~pomieszczeń. Podstawowym założeniem jest nadzorowanie, otwierania i zamykania drzwi w obrębię dużego budynku, gdzie znajduje się wiele wejść. System przechowuje dane na temat użytkowników oraz ich uprawnień dostępu, kiedy i gdzie mogą wchodzić. Informacje te znajdują się w bazie danych MySQL, umieszczonej na serwerze. Przedstawiane efekty działania systemu podzielone są na~aplikację mobilną z~system Android oraz stronę internetową. Pozostała część systemu (urządzenia sterujące, serwer), spełniają funkcje logiczne w~systemie, to znaczy łączą stricte programowo moduły pomiędzy sobą. Wirtualna część systemu oparta jest na~technologii Python 2.7 oraz Django.} 
\end{center}

 
