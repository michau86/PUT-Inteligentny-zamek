\newpage\section{Wstęp}\label{sec:wstep}

\subsection{Cel i zakres pracy}
% Odpowiedź na pytanie: Do czego system służy?
% Przedswić zadania szczegółowe z karty tematu 
	Celem utworzenia projektu oraz implementacja systemu kontroli ruchu oraz zarządzania dostępem do pomiesczeń jest zmiana sposobu zarządzania pomieszceniami w budynkach z starszych modeli opartych na fizycznych zamkach z kluczami bądź systemów opartych na kartach magnetycznych na system posługujący się urządzeniami mobilnymi. Głównymi celam naszego systemu jest usprawnienie w uzyskiwaniu dostępu do pomieszcezń dzięki wyeliminowaniu koniecznośći posiadania przy sobie wielu kluczy fizycznych oraz sytuacji w któych użytkownik zapomniał klucza lub karty magnetycznej i nie mógł uzyskać dostępu poprzez możliwość przenoszenia uprawnień między telefonami. Dodatkowo nasz projekt ma usprawniać takie elementy jak zarządzanie dostępem do wielu pomieszceń oraz kontrolą osób przebywających w dadnym poieszczenu.
W kwesti bezpieczeńśtwa systemu naszym zadaniem było spełnienie wymagania dotyczacych zabezpieczeń systemu poprzez zastosowanie szeregu funkcji kryptograficznych przy procesie uwierzytelniania jak i
przy generowaniu kluczy takich jak np. funkcje skrótu, SSH, algorytmów szyfrowania
asymetrycznego, oraz zastosowania infrastruktury klucza publicznego.

Zakres pracy w tworzeniu projektu orz implementacji obejmował takie elementy jak:
\begin{itemize}
\item Projekt i implementacja aplikacji mobilnej do zarządzania systemem od strony użytkownika oraz administratora
\item Projekt i implementacja interfejsu graficznego aplikacji mobilnej
\item Implementacja wewnętrznego PKI od strony klienta systemu
\item Projekt i implementacja strony dla administratora z podglądaem historii zamków w sieci lokalnej
\item ....
\end{itemize}
\subsection{Plan pracy}
% Spis tresci napisany słownie
Plan pracy został podzielony na trzy etapy. 
\begin{itemize}
\item Pierwszy  etap polegał na udoskonaleniu projektu który był wykonywany w ramach przedmiotu projekt zespołowy  oraz omówieniu szczegółów kluczowych wykonywanych w dalszej częśći.

\item Drugi etap polegał na implementacji danego projektu w 

\item Trzecim i ostatnim etapem było przetestowanie  działania całego systemu oraz naprawienie wykrytych błędów.
\end{itemize}
\subsection{Metodyka pracy grupowej}
Metodyka użyta  podczas pracy grupowej była oparta na metodykach zwinnych takich jak SCRUM z co 2 tygodniowymi spotkaniami omawiajacymi bieżące postępy w pracy oraz ewentualne problemy wynikające z danej specyfiki systemu oraz środowisk użytych podczas implementacji. W dalszych rozdziałach zostały szczegółowo omówione kwestie projektu implementacji oraz testowania.
